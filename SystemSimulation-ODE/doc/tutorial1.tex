\documentclass{article}
\synctex=1
\usepackage[utf8]{inputenc}
%\usepackage{babel}
\usepackage{amsmath}
\usepackage{amssymb}
\usepackage{hyperref}
\usepackage{graphicx}
\usepackage{tikz}
\usepackage{pgf}
\usepackage{xcolor}
\usepackage[a4paper]{geometry}
\usepackage{csquotes}
\usepackage{listings}
\definecolor{light-gray}{gray}{0.95}
\definecolor{dark-blue}{rgb}{0, 0, 0.5}
\definecolor{dark-red}{rgb}{0.5, 0, 0}
\definecolor{dark-green}{rgb}{0, 0.5, 0}
\lstset{language=Python, backgroundcolor = \color{light-gray}}
\lstset{
language=Python, 
backgroundcolor = \color{light-gray},
basicstyle=\small\sffamily,
stringstyle=\color{orange},
breaklines=true,
keywordstyle=\bfseries\color{dark-blue}\textit, % Schlüsselwörter fett und schwarz drucken
commentstyle=\color{dark-green}, % Kommentare blau drucken
%stringstyle=\ttfamily, % Strings im Code Schreibmaschinenähnlich - setzt sich etwas vom Code ab
showstringspaces=false} % Strings im Code ohne Kenntlichmachung von Leerzeichen - finde ich angebracht und empfehlenswert
\newtheorem{defi}{Definition}[section]

% Kommentare beginnen mit dem Zeichen "%"

% Einstellung die verhindert, dass unschöne Rahmen um Links gezogen werden
\hypersetup{
    colorlinks = true, % false bedeutet Rahmen (nicht schön)
    linkcolor={red!20!black},
    citecolor={blue!50!black},
    urlcolor={blue!80!black},
}


% Abkürzungen einführen (alphabetische Sortierung empfehlenswert):

\newcommand{\ad}{\mathrm{ad}}
% renew weil \d schon belegt ist
\renewcommand{\d}{\mathrm{d}} % aufrechtes d (für Differentialformen)
\newcommand{\NV}{{\cal N}\,}
\newcommand{\rang}{\mathrm{rang}}
\newcommand{\im}{\mathrm{im}}
\newcommand{\spann}{\mathrm{span}}
\newcommand{\R}{\mathbb{R}} % doppeltes R
\newcommand{\py}{\emph{Python}\,}
\newcommand{\scipy}{\emph{SciPy}\,}
\newcommand{\mpl}{\emph{Matplotlib}\,}
\title{Control Theory Tutorial - Car-Like Mobile Robot}
%\institution{Institute of Control Theory - Technical University Dresden, Germany}
\date{}
\author{}
% Hier beginnt das eigentliche Dokument
\begin{document}
\maketitle
\section{Introduction}
The goal of this tutorial is to teach the usage of the programming language \py as a tool for developing and simulating control systems.
\section{Model of a car-like mobile robot}
\begin{figure}[ht]
	\centering
	\def\svgwidth{0.7\textwidth}
	\input{img/car-like_mobile_robot.pdf_tex}
	\caption{Car-like mobile robot}
	\label{fig:car}
\end{figure}
Given is a nonlinear kinematic model of a car-like mobile robot, with the following system variables: position $(y_1, y_2)$ and orientation $\theta$ in the plane, the steering angle $\phi$ and the robots lateral velocity $v=\left| \vec{v} \right| $. 
\begin{subequations}\label{eq:syseq}
\begin{align}
\dot{y_1}&=v \cos (\theta) \\
\dot{y_2}&=v \sin (\theta) \\
\tan(\phi) &= \frac{l\dot{\theta}}{v}
\end{align}
\end{subequations}
To simulate this system of 1st order ordinary differential equations (ODEs), we define a state vector $\mathbf{x}=(x_1,x_2,x_3)^\mathrm{T}$ and a control vector $\mathbf{u}=(u_1,u_2)^\mathrm{T}$:
\begin{align*}
x_1 &= y_1 & u_1 = v\\
x_2 &= y_2 & u_2 = \phi \\
x_3 &= \theta 
\end{align*}
Now we can express \eqref{eq:syseq} in a general form $\dot{\mathbf{x}}=f(\mathbf{x},\mathbf{u})$:
\label{eq:ss_system}
\begin{align}
\underbrace{\begin{pmatrix} \dot{x}_1 \\ \dot{x}_2 \\ \dot{x}_3 \end{pmatrix}}_{\dot{\mathbf{x}}} = \underbrace{\begin{pmatrix}  u_1 \cos(x_3) \\ u_1 \sin(x_3) \\ \frac{1}{l}u_1 \tan(u_2) \end{pmatrix}}_{f(\mathbf{x},\mathbf{u})}
\end{align}
\section{Storing parameters}
We store the parameters of our system in a class \emph{Parameters()}.
\begin{lstlisting}
class Parameters(object):
    pass
\end{lstlisting}
We therefore create an entity of \emph{Parameters()} and assign attributes.
\begin{lstlisting}
prmtrs = Parameters() # entity of class Parameters
prmtrs.l = 0.3 # define car length
prmtrs.w = prmtrs.l*0.3 # define car width
\end{lstlisting}
\section{Simulation with SciPy's integrate package}
\label{sec:simulation}
To simulate \eqref{eq:ss_system} we need to implement the ODE system as a function in \py.
\begin{lstlisting}
def ode(x, t, prmtrs):
    """Function of the robots kinematics

    Args:
        x: state
        t: time
        prmtrs(object): parameter container class

    Returns:
        dxdt: state derivative
    """
    x1, x2, x3 = x # state vector
    u1, u2 = control(x, t) # control vector
    # dxdt = f(x, u)
    dxdt = np.array([u1 * cos(x3),
                     u1 * sin(x3),
                     1 / prmtrs.l * u1 * tan(u2)])

    # return state derivative
    return dxdt
\end{lstlisting}
In order to use $\cos(\cdot), \sin(\cdot)$ and $\tan(\cdot)$ we need to import these functions at the beginning of our code from the \emph{numpy} library.
\begin{lstlisting}
import numpy as np 
from numpy import cos, sin, tan
\end{lstlisting}
The control law is also implemented as function.
\begin{lstlisting}
def control(x, t):
    """Function of the control law

    Args:
        x: state vector
        t: time

    Returns:
        u: control vector

    """
    u = [1, 0.25] # v, phi
    
    return u
\end{lstlisting}
As a first simple heuristic, we set $(u_1, u_2)$ equal to constant values. Later we can implement an arbitrary function, for expample a feedback law $\mathbf{u}=k(\mathbf{x})$.

\subsection{Solution of the initial value problem (IVP) using \scipy}
To simulate \eqref{eq:ss_system} we need to solve an IVP. In \py we can use the library \scipy and its sub-package \emph{integrate}, which delivers different solvers for IVPs.
\begin{lstlisting}
from scipy.integrate import odeint
\end{lstlisting}
We then define the simulation time and the initial state value.
\begin{lstlisting}
t0 = 0 # start 
tend = 10 # end
dt = 0.01 # stepsize (not of the solver, just evaluation points)
tt = np.arange(t0, tend, dt) # simulation interval

x0 = [0, 0, 0] # initial state value
\end{lstlisting}
Now we can parse these parameters and our ODE function to the solver.
\begin{lstlisting}
x_traj = odeint(ode, x0, tt, args=(prmtrs, )) # solution of the IVP
\end{lstlisting}
The output is an array of size length(tt)$\times$lenght($\mathbf{x}$).
\section{Plotting using \mpl}
\label{sec:plot}
For plotting the output of our simulation, we use the library \mpl and its sub-package \emph{pyplot}, which delivers a user experience similar to \emph{MATLAB}.
\begin{lstlisting}
import matplotlib.pyplot as plt
\end{lstlisting}
We encase our plotting instructions in a function. This way, we can define parameters of our plot, which we would like to change easily, for example figure width, or if the figure should be saved on the hard drive.
\begin{lstlisting}
def plot_data(fig_width, fig_height, save=False):
    """Plotting function of simulated state and actions

    Args:
        fig_width: figure width in cm
        fig_height: figure height in cm
        save (bool) : save figure (default: False)

    Returns: None

    """
    # creating a figure with 2 subplots, that share the x-axis
    fig1, (ax1, ax2) = plt.subplots(2, sharex=True)

    # set figure size to desired values
    fig1.set_size_inches(fig_width / 2.54, fig_height / 2.54)

    # plot y_1 in subplot 1
    ax1.plot(tt, x_traj[:, 0], label='$y_1(t)$', lw=1, color='r')

    # plot y_2 in subplot 1
    ax1.plot(tt, x_traj[:, 1], label='$y_2(t)$', lw=1, color='b')

    # plot theta in subplot 2
    ax2.plot(tt, x_traj[:, 2], label=r'$\theta(t)$', lw=1, color='g')

    ax1.grid(True)
    ax2.grid(True)
    # set the labels on the x and y axis in subplot 1
    ax1.set_ylabel(r'm')
    ax1.set_xlabel(r't in s')
    ax2.set_ylabel(r'rad')
    ax2.set_xlabel(r't in s')

    # put a legend in the plot
    ax1.legend()
    ax2.legend()

    #automatically adjusts subplot to fit in figure window
    plt.tight_layout()

    # save the figure in the working directory
    if save:
        plt.savefig('state_trajectory.pdf')  # save output as pdf
        plt.savefig('state_trajectory.pgf')  # for easy export to LaTex
    return None
\end{lstlisting}
Finally, we have to execute
\begin{lstlisting}
plt.show()
\end{lstlisting}
to display the results. If your not satisfied with the result, you can change other properties of the plot, like linewidth or -color and many others easily. Just look up the documentation of \mpl : \url{https://matplotlib.org/index.html}
\begin{figure}[h]
\label{fig:state_traj}
   \centering      
   %% Creator: Matplotlib, PGF backend
%%
%% To include the figure in your LaTeX document, write
%%   \input{<filename>.pgf}
%%
%% Make sure the required packages are loaded in your preamble
%%   \usepackage{pgf}
%%
%% Figures using additional raster images can only be included by \input if
%% they are in the same directory as the main LaTeX file. For loading figures
%% from other directories you can use the `import` package
%%   \usepackage{import}
%% and then include the figures with
%%   \import{<path to file>}{<filename>.pgf}
%%
%% Matplotlib used the following preamble
%%   \usepackage{fontspec}
%%   \setmainfont{DejaVu Serif}
%%   \setsansfont{DejaVu Sans}
%%   \setmonofont{DejaVu Sans Mono}
%%
\begingroup%
\makeatletter%
\begin{pgfpicture}%
\pgfpathrectangle{\pgfpointorigin}{\pgfqpoint{4.724409in}{6.299213in}}%
\pgfusepath{use as bounding box, clip}%
\begin{pgfscope}%
\pgfsetbuttcap%
\pgfsetmiterjoin%
\definecolor{currentfill}{rgb}{1.000000,1.000000,1.000000}%
\pgfsetfillcolor{currentfill}%
\pgfsetlinewidth{0.000000pt}%
\definecolor{currentstroke}{rgb}{1.000000,1.000000,1.000000}%
\pgfsetstrokecolor{currentstroke}%
\pgfsetdash{}{0pt}%
\pgfpathmoveto{\pgfqpoint{0.000000in}{0.000000in}}%
\pgfpathlineto{\pgfqpoint{4.724409in}{0.000000in}}%
\pgfpathlineto{\pgfqpoint{4.724409in}{6.299213in}}%
\pgfpathlineto{\pgfqpoint{0.000000in}{6.299213in}}%
\pgfpathclose%
\pgfusepath{fill}%
\end{pgfscope}%
\begin{pgfscope}%
\pgfsetbuttcap%
\pgfsetmiterjoin%
\definecolor{currentfill}{rgb}{1.000000,1.000000,1.000000}%
\pgfsetfillcolor{currentfill}%
\pgfsetlinewidth{0.000000pt}%
\definecolor{currentstroke}{rgb}{0.000000,0.000000,0.000000}%
\pgfsetstrokecolor{currentstroke}%
\pgfsetstrokeopacity{0.000000}%
\pgfsetdash{}{0pt}%
\pgfpathmoveto{\pgfqpoint{0.706528in}{4.458920in}}%
\pgfpathlineto{\pgfqpoint{3.972882in}{4.458920in}}%
\pgfpathlineto{\pgfqpoint{3.972882in}{5.945879in}}%
\pgfpathlineto{\pgfqpoint{0.706528in}{5.945879in}}%
\pgfpathclose%
\pgfusepath{fill}%
\end{pgfscope}%
\begin{pgfscope}%
\pgfpathrectangle{\pgfqpoint{0.706528in}{4.458920in}}{\pgfqpoint{3.266354in}{1.486960in}} %
\pgfusepath{clip}%
\pgfsetrectcap%
\pgfsetroundjoin%
\pgfsetlinewidth{0.803000pt}%
\definecolor{currentstroke}{rgb}{0.690196,0.690196,0.690196}%
\pgfsetstrokecolor{currentstroke}%
\pgfsetdash{}{0pt}%
\pgfpathmoveto{\pgfqpoint{0.854998in}{4.458920in}}%
\pgfpathlineto{\pgfqpoint{0.854998in}{5.945879in}}%
\pgfusepath{stroke}%
\end{pgfscope}%
\begin{pgfscope}%
\pgfsetbuttcap%
\pgfsetroundjoin%
\definecolor{currentfill}{rgb}{0.000000,0.000000,0.000000}%
\pgfsetfillcolor{currentfill}%
\pgfsetlinewidth{0.803000pt}%
\definecolor{currentstroke}{rgb}{0.000000,0.000000,0.000000}%
\pgfsetstrokecolor{currentstroke}%
\pgfsetdash{}{0pt}%
\pgfsys@defobject{currentmarker}{\pgfqpoint{0.000000in}{-0.048611in}}{\pgfqpoint{0.000000in}{0.000000in}}{%
\pgfpathmoveto{\pgfqpoint{0.000000in}{0.000000in}}%
\pgfpathlineto{\pgfqpoint{0.000000in}{-0.048611in}}%
\pgfusepath{stroke,fill}%
}%
\begin{pgfscope}%
\pgfsys@transformshift{0.854998in}{4.458920in}%
\pgfsys@useobject{currentmarker}{}%
\end{pgfscope}%
\end{pgfscope}%
\begin{pgfscope}%
\pgfpathrectangle{\pgfqpoint{0.706528in}{4.458920in}}{\pgfqpoint{3.266354in}{1.486960in}} %
\pgfusepath{clip}%
\pgfsetrectcap%
\pgfsetroundjoin%
\pgfsetlinewidth{0.803000pt}%
\definecolor{currentstroke}{rgb}{0.690196,0.690196,0.690196}%
\pgfsetstrokecolor{currentstroke}%
\pgfsetdash{}{0pt}%
\pgfpathmoveto{\pgfqpoint{1.448881in}{4.458920in}}%
\pgfpathlineto{\pgfqpoint{1.448881in}{5.945879in}}%
\pgfusepath{stroke}%
\end{pgfscope}%
\begin{pgfscope}%
\pgfsetbuttcap%
\pgfsetroundjoin%
\definecolor{currentfill}{rgb}{0.000000,0.000000,0.000000}%
\pgfsetfillcolor{currentfill}%
\pgfsetlinewidth{0.803000pt}%
\definecolor{currentstroke}{rgb}{0.000000,0.000000,0.000000}%
\pgfsetstrokecolor{currentstroke}%
\pgfsetdash{}{0pt}%
\pgfsys@defobject{currentmarker}{\pgfqpoint{0.000000in}{-0.048611in}}{\pgfqpoint{0.000000in}{0.000000in}}{%
\pgfpathmoveto{\pgfqpoint{0.000000in}{0.000000in}}%
\pgfpathlineto{\pgfqpoint{0.000000in}{-0.048611in}}%
\pgfusepath{stroke,fill}%
}%
\begin{pgfscope}%
\pgfsys@transformshift{1.448881in}{4.458920in}%
\pgfsys@useobject{currentmarker}{}%
\end{pgfscope}%
\end{pgfscope}%
\begin{pgfscope}%
\pgfpathrectangle{\pgfqpoint{0.706528in}{4.458920in}}{\pgfqpoint{3.266354in}{1.486960in}} %
\pgfusepath{clip}%
\pgfsetrectcap%
\pgfsetroundjoin%
\pgfsetlinewidth{0.803000pt}%
\definecolor{currentstroke}{rgb}{0.690196,0.690196,0.690196}%
\pgfsetstrokecolor{currentstroke}%
\pgfsetdash{}{0pt}%
\pgfpathmoveto{\pgfqpoint{2.042763in}{4.458920in}}%
\pgfpathlineto{\pgfqpoint{2.042763in}{5.945879in}}%
\pgfusepath{stroke}%
\end{pgfscope}%
\begin{pgfscope}%
\pgfsetbuttcap%
\pgfsetroundjoin%
\definecolor{currentfill}{rgb}{0.000000,0.000000,0.000000}%
\pgfsetfillcolor{currentfill}%
\pgfsetlinewidth{0.803000pt}%
\definecolor{currentstroke}{rgb}{0.000000,0.000000,0.000000}%
\pgfsetstrokecolor{currentstroke}%
\pgfsetdash{}{0pt}%
\pgfsys@defobject{currentmarker}{\pgfqpoint{0.000000in}{-0.048611in}}{\pgfqpoint{0.000000in}{0.000000in}}{%
\pgfpathmoveto{\pgfqpoint{0.000000in}{0.000000in}}%
\pgfpathlineto{\pgfqpoint{0.000000in}{-0.048611in}}%
\pgfusepath{stroke,fill}%
}%
\begin{pgfscope}%
\pgfsys@transformshift{2.042763in}{4.458920in}%
\pgfsys@useobject{currentmarker}{}%
\end{pgfscope}%
\end{pgfscope}%
\begin{pgfscope}%
\pgfpathrectangle{\pgfqpoint{0.706528in}{4.458920in}}{\pgfqpoint{3.266354in}{1.486960in}} %
\pgfusepath{clip}%
\pgfsetrectcap%
\pgfsetroundjoin%
\pgfsetlinewidth{0.803000pt}%
\definecolor{currentstroke}{rgb}{0.690196,0.690196,0.690196}%
\pgfsetstrokecolor{currentstroke}%
\pgfsetdash{}{0pt}%
\pgfpathmoveto{\pgfqpoint{2.636646in}{4.458920in}}%
\pgfpathlineto{\pgfqpoint{2.636646in}{5.945879in}}%
\pgfusepath{stroke}%
\end{pgfscope}%
\begin{pgfscope}%
\pgfsetbuttcap%
\pgfsetroundjoin%
\definecolor{currentfill}{rgb}{0.000000,0.000000,0.000000}%
\pgfsetfillcolor{currentfill}%
\pgfsetlinewidth{0.803000pt}%
\definecolor{currentstroke}{rgb}{0.000000,0.000000,0.000000}%
\pgfsetstrokecolor{currentstroke}%
\pgfsetdash{}{0pt}%
\pgfsys@defobject{currentmarker}{\pgfqpoint{0.000000in}{-0.048611in}}{\pgfqpoint{0.000000in}{0.000000in}}{%
\pgfpathmoveto{\pgfqpoint{0.000000in}{0.000000in}}%
\pgfpathlineto{\pgfqpoint{0.000000in}{-0.048611in}}%
\pgfusepath{stroke,fill}%
}%
\begin{pgfscope}%
\pgfsys@transformshift{2.636646in}{4.458920in}%
\pgfsys@useobject{currentmarker}{}%
\end{pgfscope}%
\end{pgfscope}%
\begin{pgfscope}%
\pgfpathrectangle{\pgfqpoint{0.706528in}{4.458920in}}{\pgfqpoint{3.266354in}{1.486960in}} %
\pgfusepath{clip}%
\pgfsetrectcap%
\pgfsetroundjoin%
\pgfsetlinewidth{0.803000pt}%
\definecolor{currentstroke}{rgb}{0.690196,0.690196,0.690196}%
\pgfsetstrokecolor{currentstroke}%
\pgfsetdash{}{0pt}%
\pgfpathmoveto{\pgfqpoint{3.230529in}{4.458920in}}%
\pgfpathlineto{\pgfqpoint{3.230529in}{5.945879in}}%
\pgfusepath{stroke}%
\end{pgfscope}%
\begin{pgfscope}%
\pgfsetbuttcap%
\pgfsetroundjoin%
\definecolor{currentfill}{rgb}{0.000000,0.000000,0.000000}%
\pgfsetfillcolor{currentfill}%
\pgfsetlinewidth{0.803000pt}%
\definecolor{currentstroke}{rgb}{0.000000,0.000000,0.000000}%
\pgfsetstrokecolor{currentstroke}%
\pgfsetdash{}{0pt}%
\pgfsys@defobject{currentmarker}{\pgfqpoint{0.000000in}{-0.048611in}}{\pgfqpoint{0.000000in}{0.000000in}}{%
\pgfpathmoveto{\pgfqpoint{0.000000in}{0.000000in}}%
\pgfpathlineto{\pgfqpoint{0.000000in}{-0.048611in}}%
\pgfusepath{stroke,fill}%
}%
\begin{pgfscope}%
\pgfsys@transformshift{3.230529in}{4.458920in}%
\pgfsys@useobject{currentmarker}{}%
\end{pgfscope}%
\end{pgfscope}%
\begin{pgfscope}%
\pgfpathrectangle{\pgfqpoint{0.706528in}{4.458920in}}{\pgfqpoint{3.266354in}{1.486960in}} %
\pgfusepath{clip}%
\pgfsetrectcap%
\pgfsetroundjoin%
\pgfsetlinewidth{0.803000pt}%
\definecolor{currentstroke}{rgb}{0.690196,0.690196,0.690196}%
\pgfsetstrokecolor{currentstroke}%
\pgfsetdash{}{0pt}%
\pgfpathmoveto{\pgfqpoint{3.824411in}{4.458920in}}%
\pgfpathlineto{\pgfqpoint{3.824411in}{5.945879in}}%
\pgfusepath{stroke}%
\end{pgfscope}%
\begin{pgfscope}%
\pgfsetbuttcap%
\pgfsetroundjoin%
\definecolor{currentfill}{rgb}{0.000000,0.000000,0.000000}%
\pgfsetfillcolor{currentfill}%
\pgfsetlinewidth{0.803000pt}%
\definecolor{currentstroke}{rgb}{0.000000,0.000000,0.000000}%
\pgfsetstrokecolor{currentstroke}%
\pgfsetdash{}{0pt}%
\pgfsys@defobject{currentmarker}{\pgfqpoint{0.000000in}{-0.048611in}}{\pgfqpoint{0.000000in}{0.000000in}}{%
\pgfpathmoveto{\pgfqpoint{0.000000in}{0.000000in}}%
\pgfpathlineto{\pgfqpoint{0.000000in}{-0.048611in}}%
\pgfusepath{stroke,fill}%
}%
\begin{pgfscope}%
\pgfsys@transformshift{3.824411in}{4.458920in}%
\pgfsys@useobject{currentmarker}{}%
\end{pgfscope}%
\end{pgfscope}%
\begin{pgfscope}%
\pgftext[x=2.339705in,y=4.403364in,,top]{\sffamily\fontsize{10.000000}{12.000000}\selectfont t in s}%
\end{pgfscope}%
\begin{pgfscope}%
\pgfpathrectangle{\pgfqpoint{0.706528in}{4.458920in}}{\pgfqpoint{3.266354in}{1.486960in}} %
\pgfusepath{clip}%
\pgfsetrectcap%
\pgfsetroundjoin%
\pgfsetlinewidth{0.803000pt}%
\definecolor{currentstroke}{rgb}{0.690196,0.690196,0.690196}%
\pgfsetstrokecolor{currentstroke}%
\pgfsetdash{}{0pt}%
\pgfpathmoveto{\pgfqpoint{0.706528in}{4.526509in}}%
\pgfpathlineto{\pgfqpoint{3.972882in}{4.526509in}}%
\pgfusepath{stroke}%
\end{pgfscope}%
\begin{pgfscope}%
\pgfsetbuttcap%
\pgfsetroundjoin%
\definecolor{currentfill}{rgb}{0.000000,0.000000,0.000000}%
\pgfsetfillcolor{currentfill}%
\pgfsetlinewidth{0.803000pt}%
\definecolor{currentstroke}{rgb}{0.000000,0.000000,0.000000}%
\pgfsetstrokecolor{currentstroke}%
\pgfsetdash{}{0pt}%
\pgfsys@defobject{currentmarker}{\pgfqpoint{-0.048611in}{0.000000in}}{\pgfqpoint{0.000000in}{0.000000in}}{%
\pgfpathmoveto{\pgfqpoint{0.000000in}{0.000000in}}%
\pgfpathlineto{\pgfqpoint{-0.048611in}{0.000000in}}%
\pgfusepath{stroke,fill}%
}%
\begin{pgfscope}%
\pgfsys@transformshift{0.706528in}{4.526509in}%
\pgfsys@useobject{currentmarker}{}%
\end{pgfscope}%
\end{pgfscope}%
\begin{pgfscope}%
\pgftext[x=0.410670in,y=4.473747in,left,base]{\sffamily\fontsize{10.000000}{12.000000}\selectfont 0.0}%
\end{pgfscope}%
\begin{pgfscope}%
\pgfpathrectangle{\pgfqpoint{0.706528in}{4.458920in}}{\pgfqpoint{3.266354in}{1.486960in}} %
\pgfusepath{clip}%
\pgfsetrectcap%
\pgfsetroundjoin%
\pgfsetlinewidth{0.803000pt}%
\definecolor{currentstroke}{rgb}{0.690196,0.690196,0.690196}%
\pgfsetstrokecolor{currentstroke}%
\pgfsetdash{}{0pt}%
\pgfpathmoveto{\pgfqpoint{0.706528in}{5.005494in}}%
\pgfpathlineto{\pgfqpoint{3.972882in}{5.005494in}}%
\pgfusepath{stroke}%
\end{pgfscope}%
\begin{pgfscope}%
\pgfsetbuttcap%
\pgfsetroundjoin%
\definecolor{currentfill}{rgb}{0.000000,0.000000,0.000000}%
\pgfsetfillcolor{currentfill}%
\pgfsetlinewidth{0.803000pt}%
\definecolor{currentstroke}{rgb}{0.000000,0.000000,0.000000}%
\pgfsetstrokecolor{currentstroke}%
\pgfsetdash{}{0pt}%
\pgfsys@defobject{currentmarker}{\pgfqpoint{-0.048611in}{0.000000in}}{\pgfqpoint{0.000000in}{0.000000in}}{%
\pgfpathmoveto{\pgfqpoint{0.000000in}{0.000000in}}%
\pgfpathlineto{\pgfqpoint{-0.048611in}{0.000000in}}%
\pgfusepath{stroke,fill}%
}%
\begin{pgfscope}%
\pgfsys@transformshift{0.706528in}{5.005494in}%
\pgfsys@useobject{currentmarker}{}%
\end{pgfscope}%
\end{pgfscope}%
\begin{pgfscope}%
\pgftext[x=0.410670in,y=4.952733in,left,base]{\sffamily\fontsize{10.000000}{12.000000}\selectfont 0.5}%
\end{pgfscope}%
\begin{pgfscope}%
\pgfpathrectangle{\pgfqpoint{0.706528in}{4.458920in}}{\pgfqpoint{3.266354in}{1.486960in}} %
\pgfusepath{clip}%
\pgfsetrectcap%
\pgfsetroundjoin%
\pgfsetlinewidth{0.803000pt}%
\definecolor{currentstroke}{rgb}{0.690196,0.690196,0.690196}%
\pgfsetstrokecolor{currentstroke}%
\pgfsetdash{}{0pt}%
\pgfpathmoveto{\pgfqpoint{0.706528in}{5.484480in}}%
\pgfpathlineto{\pgfqpoint{3.972882in}{5.484480in}}%
\pgfusepath{stroke}%
\end{pgfscope}%
\begin{pgfscope}%
\pgfsetbuttcap%
\pgfsetroundjoin%
\definecolor{currentfill}{rgb}{0.000000,0.000000,0.000000}%
\pgfsetfillcolor{currentfill}%
\pgfsetlinewidth{0.803000pt}%
\definecolor{currentstroke}{rgb}{0.000000,0.000000,0.000000}%
\pgfsetstrokecolor{currentstroke}%
\pgfsetdash{}{0pt}%
\pgfsys@defobject{currentmarker}{\pgfqpoint{-0.048611in}{0.000000in}}{\pgfqpoint{0.000000in}{0.000000in}}{%
\pgfpathmoveto{\pgfqpoint{0.000000in}{0.000000in}}%
\pgfpathlineto{\pgfqpoint{-0.048611in}{0.000000in}}%
\pgfusepath{stroke,fill}%
}%
\begin{pgfscope}%
\pgfsys@transformshift{0.706528in}{5.484480in}%
\pgfsys@useobject{currentmarker}{}%
\end{pgfscope}%
\end{pgfscope}%
\begin{pgfscope}%
\pgftext[x=0.410670in,y=5.431719in,left,base]{\sffamily\fontsize{10.000000}{12.000000}\selectfont 1.0}%
\end{pgfscope}%
\begin{pgfscope}%
\pgftext[x=0.355114in,y=5.202399in,,bottom,rotate=90.000000]{\sffamily\fontsize{10.000000}{12.000000}\selectfont m}%
\end{pgfscope}%
\begin{pgfscope}%
\pgfpathrectangle{\pgfqpoint{0.706528in}{4.458920in}}{\pgfqpoint{3.266354in}{1.486960in}} %
\pgfusepath{clip}%
\pgfsetrectcap%
\pgfsetroundjoin%
\pgfsetlinewidth{1.003750pt}%
\definecolor{currentstroke}{rgb}{1.000000,0.000000,0.000000}%
\pgfsetstrokecolor{currentstroke}%
\pgfsetdash{}{0pt}%
\pgfpathmoveto{\pgfqpoint{0.854998in}{4.526509in}}%
\pgfpathlineto{\pgfqpoint{0.926264in}{4.639619in}}%
\pgfpathlineto{\pgfqpoint{0.997530in}{4.748338in}}%
\pgfpathlineto{\pgfqpoint{1.056918in}{4.834933in}}%
\pgfpathlineto{\pgfqpoint{1.116307in}{4.917440in}}%
\pgfpathlineto{\pgfqpoint{1.175695in}{4.995532in}}%
\pgfpathlineto{\pgfqpoint{1.223206in}{5.054661in}}%
\pgfpathlineto{\pgfqpoint{1.270716in}{5.110719in}}%
\pgfpathlineto{\pgfqpoint{1.318227in}{5.163647in}}%
\pgfpathlineto{\pgfqpoint{1.365737in}{5.213412in}}%
\pgfpathlineto{\pgfqpoint{1.413248in}{5.260007in}}%
\pgfpathlineto{\pgfqpoint{1.460759in}{5.303449in}}%
\pgfpathlineto{\pgfqpoint{1.508269in}{5.343773in}}%
\pgfpathlineto{\pgfqpoint{1.555780in}{5.381035in}}%
\pgfpathlineto{\pgfqpoint{1.603290in}{5.415304in}}%
\pgfpathlineto{\pgfqpoint{1.650801in}{5.446666in}}%
\pgfpathlineto{\pgfqpoint{1.698312in}{5.475218in}}%
\pgfpathlineto{\pgfqpoint{1.745822in}{5.501068in}}%
\pgfpathlineto{\pgfqpoint{1.793333in}{5.524331in}}%
\pgfpathlineto{\pgfqpoint{1.840843in}{5.545131in}}%
\pgfpathlineto{\pgfqpoint{1.888354in}{5.563597in}}%
\pgfpathlineto{\pgfqpoint{1.935865in}{5.579860in}}%
\pgfpathlineto{\pgfqpoint{1.983375in}{5.594055in}}%
\pgfpathlineto{\pgfqpoint{2.030886in}{5.606320in}}%
\pgfpathlineto{\pgfqpoint{2.090274in}{5.619144in}}%
\pgfpathlineto{\pgfqpoint{2.149662in}{5.629431in}}%
\pgfpathlineto{\pgfqpoint{2.209051in}{5.637443in}}%
\pgfpathlineto{\pgfqpoint{2.268439in}{5.643435in}}%
\pgfpathlineto{\pgfqpoint{2.339705in}{5.648307in}}%
\pgfpathlineto{\pgfqpoint{2.410971in}{5.651036in}}%
\pgfpathlineto{\pgfqpoint{2.494114in}{5.652025in}}%
\pgfpathlineto{\pgfqpoint{2.589135in}{5.650952in}}%
\pgfpathlineto{\pgfqpoint{2.719790in}{5.646976in}}%
\pgfpathlineto{\pgfqpoint{3.183018in}{5.630432in}}%
\pgfpathlineto{\pgfqpoint{3.313672in}{5.629063in}}%
\pgfpathlineto{\pgfqpoint{3.824411in}{5.629047in}}%
\pgfpathlineto{\pgfqpoint{3.824411in}{5.629047in}}%
\pgfusepath{stroke}%
\end{pgfscope}%
\begin{pgfscope}%
\pgfpathrectangle{\pgfqpoint{0.706528in}{4.458920in}}{\pgfqpoint{3.266354in}{1.486960in}} %
\pgfusepath{clip}%
\pgfsetrectcap%
\pgfsetroundjoin%
\pgfsetlinewidth{1.003750pt}%
\definecolor{currentstroke}{rgb}{0.000000,0.000000,1.000000}%
\pgfsetstrokecolor{currentstroke}%
\pgfsetdash{}{0pt}%
\pgfpathmoveto{\pgfqpoint{0.854998in}{4.526509in}}%
\pgfpathlineto{\pgfqpoint{0.890631in}{4.527955in}}%
\pgfpathlineto{\pgfqpoint{0.926264in}{4.532207in}}%
\pgfpathlineto{\pgfqpoint{0.961897in}{4.539129in}}%
\pgfpathlineto{\pgfqpoint{0.997530in}{4.548586in}}%
\pgfpathlineto{\pgfqpoint{1.033163in}{4.560434in}}%
\pgfpathlineto{\pgfqpoint{1.068796in}{4.574533in}}%
\pgfpathlineto{\pgfqpoint{1.104429in}{4.590737in}}%
\pgfpathlineto{\pgfqpoint{1.140062in}{4.608904in}}%
\pgfpathlineto{\pgfqpoint{1.187573in}{4.635932in}}%
\pgfpathlineto{\pgfqpoint{1.235083in}{4.665857in}}%
\pgfpathlineto{\pgfqpoint{1.282594in}{4.698349in}}%
\pgfpathlineto{\pgfqpoint{1.341982in}{4.742085in}}%
\pgfpathlineto{\pgfqpoint{1.401370in}{4.788722in}}%
\pgfpathlineto{\pgfqpoint{1.472636in}{4.847707in}}%
\pgfpathlineto{\pgfqpoint{1.555780in}{4.919514in}}%
\pgfpathlineto{\pgfqpoint{1.686434in}{5.035791in}}%
\pgfpathlineto{\pgfqpoint{1.864599in}{5.194077in}}%
\pgfpathlineto{\pgfqpoint{1.959620in}{5.275588in}}%
\pgfpathlineto{\pgfqpoint{2.042763in}{5.344159in}}%
\pgfpathlineto{\pgfqpoint{2.114029in}{5.400450in}}%
\pgfpathlineto{\pgfqpoint{2.185295in}{5.454152in}}%
\pgfpathlineto{\pgfqpoint{2.256561in}{5.505040in}}%
\pgfpathlineto{\pgfqpoint{2.327827in}{5.552937in}}%
\pgfpathlineto{\pgfqpoint{2.399093in}{5.597709in}}%
\pgfpathlineto{\pgfqpoint{2.458481in}{5.632559in}}%
\pgfpathlineto{\pgfqpoint{2.517869in}{5.665128in}}%
\pgfpathlineto{\pgfqpoint{2.577258in}{5.695390in}}%
\pgfpathlineto{\pgfqpoint{2.636646in}{5.723326in}}%
\pgfpathlineto{\pgfqpoint{2.696034in}{5.748927in}}%
\pgfpathlineto{\pgfqpoint{2.755422in}{5.772192in}}%
\pgfpathlineto{\pgfqpoint{2.814811in}{5.793125in}}%
\pgfpathlineto{\pgfqpoint{2.874199in}{5.811732in}}%
\pgfpathlineto{\pgfqpoint{2.933587in}{5.828025in}}%
\pgfpathlineto{\pgfqpoint{2.992976in}{5.842014in}}%
\pgfpathlineto{\pgfqpoint{3.052364in}{5.853712in}}%
\pgfpathlineto{\pgfqpoint{3.111752in}{5.863129in}}%
\pgfpathlineto{\pgfqpoint{3.171140in}{5.870276in}}%
\pgfpathlineto{\pgfqpoint{3.230529in}{5.875161in}}%
\pgfpathlineto{\pgfqpoint{3.289917in}{5.877789in}}%
\pgfpathlineto{\pgfqpoint{3.361183in}{5.878290in}}%
\pgfpathlineto{\pgfqpoint{3.824411in}{5.878290in}}%
\pgfpathlineto{\pgfqpoint{3.824411in}{5.878290in}}%
\pgfusepath{stroke}%
\end{pgfscope}%
\begin{pgfscope}%
\pgfsetrectcap%
\pgfsetmiterjoin%
\pgfsetlinewidth{0.803000pt}%
\definecolor{currentstroke}{rgb}{0.000000,0.000000,0.000000}%
\pgfsetstrokecolor{currentstroke}%
\pgfsetdash{}{0pt}%
\pgfpathmoveto{\pgfqpoint{0.706528in}{4.458920in}}%
\pgfpathlineto{\pgfqpoint{0.706528in}{5.945879in}}%
\pgfusepath{stroke}%
\end{pgfscope}%
\begin{pgfscope}%
\pgfsetrectcap%
\pgfsetmiterjoin%
\pgfsetlinewidth{0.803000pt}%
\definecolor{currentstroke}{rgb}{0.000000,0.000000,0.000000}%
\pgfsetstrokecolor{currentstroke}%
\pgfsetdash{}{0pt}%
\pgfpathmoveto{\pgfqpoint{3.972882in}{4.458920in}}%
\pgfpathlineto{\pgfqpoint{3.972882in}{5.945879in}}%
\pgfusepath{stroke}%
\end{pgfscope}%
\begin{pgfscope}%
\pgfsetrectcap%
\pgfsetmiterjoin%
\pgfsetlinewidth{0.803000pt}%
\definecolor{currentstroke}{rgb}{0.000000,0.000000,0.000000}%
\pgfsetstrokecolor{currentstroke}%
\pgfsetdash{}{0pt}%
\pgfpathmoveto{\pgfqpoint{0.706528in}{4.458920in}}%
\pgfpathlineto{\pgfqpoint{3.972882in}{4.458920in}}%
\pgfusepath{stroke}%
\end{pgfscope}%
\begin{pgfscope}%
\pgfsetrectcap%
\pgfsetmiterjoin%
\pgfsetlinewidth{0.803000pt}%
\definecolor{currentstroke}{rgb}{0.000000,0.000000,0.000000}%
\pgfsetstrokecolor{currentstroke}%
\pgfsetdash{}{0pt}%
\pgfpathmoveto{\pgfqpoint{0.706528in}{5.945879in}}%
\pgfpathlineto{\pgfqpoint{3.972882in}{5.945879in}}%
\pgfusepath{stroke}%
\end{pgfscope}%
\begin{pgfscope}%
\pgftext[x=2.339705in,y=6.029213in,,base]{\sffamily\fontsize{12.000000}{14.400000}\selectfont Position coordinates}%
\end{pgfscope}%
\begin{pgfscope}%
\pgfsetbuttcap%
\pgfsetmiterjoin%
\definecolor{currentfill}{rgb}{1.000000,1.000000,1.000000}%
\pgfsetfillcolor{currentfill}%
\pgfsetfillopacity{0.800000}%
\pgfsetlinewidth{1.003750pt}%
\definecolor{currentstroke}{rgb}{0.800000,0.800000,0.800000}%
\pgfsetstrokecolor{currentstroke}%
\pgfsetstrokeopacity{0.800000}%
\pgfsetdash{}{0pt}%
\pgfpathmoveto{\pgfqpoint{0.803750in}{5.415389in}}%
\pgfpathlineto{\pgfqpoint{1.536776in}{5.415389in}}%
\pgfpathquadraticcurveto{\pgfqpoint{1.564554in}{5.415389in}}{\pgfqpoint{1.564554in}{5.443166in}}%
\pgfpathlineto{\pgfqpoint{1.564554in}{5.848657in}}%
\pgfpathquadraticcurveto{\pgfqpoint{1.564554in}{5.876435in}}{\pgfqpoint{1.536776in}{5.876435in}}%
\pgfpathlineto{\pgfqpoint{0.803750in}{5.876435in}}%
\pgfpathquadraticcurveto{\pgfqpoint{0.775972in}{5.876435in}}{\pgfqpoint{0.775972in}{5.848657in}}%
\pgfpathlineto{\pgfqpoint{0.775972in}{5.443166in}}%
\pgfpathquadraticcurveto{\pgfqpoint{0.775972in}{5.415389in}}{\pgfqpoint{0.803750in}{5.415389in}}%
\pgfpathclose%
\pgfusepath{stroke,fill}%
\end{pgfscope}%
\begin{pgfscope}%
\pgfsetrectcap%
\pgfsetroundjoin%
\pgfsetlinewidth{1.003750pt}%
\definecolor{currentstroke}{rgb}{1.000000,0.000000,0.000000}%
\pgfsetstrokecolor{currentstroke}%
\pgfsetdash{}{0pt}%
\pgfpathmoveto{\pgfqpoint{0.831528in}{5.763967in}}%
\pgfpathlineto{\pgfqpoint{1.109306in}{5.763967in}}%
\pgfusepath{stroke}%
\end{pgfscope}%
\begin{pgfscope}%
\pgftext[x=1.220417in,y=5.715356in,left,base]{\sffamily\fontsize{10.000000}{12.000000}\selectfont \(\displaystyle y_1(t)\)}%
\end{pgfscope}%
\begin{pgfscope}%
\pgfsetrectcap%
\pgfsetroundjoin%
\pgfsetlinewidth{1.003750pt}%
\definecolor{currentstroke}{rgb}{0.000000,0.000000,1.000000}%
\pgfsetstrokecolor{currentstroke}%
\pgfsetdash{}{0pt}%
\pgfpathmoveto{\pgfqpoint{0.831528in}{5.554278in}}%
\pgfpathlineto{\pgfqpoint{1.109306in}{5.554278in}}%
\pgfusepath{stroke}%
\end{pgfscope}%
\begin{pgfscope}%
\pgftext[x=1.220417in,y=5.505666in,left,base]{\sffamily\fontsize{10.000000}{12.000000}\selectfont \(\displaystyle y_2(t)\)}%
\end{pgfscope}%
\begin{pgfscope}%
\pgfsetbuttcap%
\pgfsetmiterjoin%
\definecolor{currentfill}{rgb}{1.000000,1.000000,1.000000}%
\pgfsetfillcolor{currentfill}%
\pgfsetlinewidth{0.000000pt}%
\definecolor{currentstroke}{rgb}{0.000000,0.000000,0.000000}%
\pgfsetstrokecolor{currentstroke}%
\pgfsetstrokeopacity{0.000000}%
\pgfsetdash{}{0pt}%
\pgfpathmoveto{\pgfqpoint{0.706528in}{2.423071in}}%
\pgfpathlineto{\pgfqpoint{3.972882in}{2.423071in}}%
\pgfpathlineto{\pgfqpoint{3.972882in}{3.910031in}}%
\pgfpathlineto{\pgfqpoint{0.706528in}{3.910031in}}%
\pgfpathclose%
\pgfusepath{fill}%
\end{pgfscope}%
\begin{pgfscope}%
\pgfpathrectangle{\pgfqpoint{0.706528in}{2.423071in}}{\pgfqpoint{3.266354in}{1.486960in}} %
\pgfusepath{clip}%
\pgfsetrectcap%
\pgfsetroundjoin%
\pgfsetlinewidth{0.803000pt}%
\definecolor{currentstroke}{rgb}{0.690196,0.690196,0.690196}%
\pgfsetstrokecolor{currentstroke}%
\pgfsetdash{}{0pt}%
\pgfpathmoveto{\pgfqpoint{0.854998in}{2.423071in}}%
\pgfpathlineto{\pgfqpoint{0.854998in}{3.910031in}}%
\pgfusepath{stroke}%
\end{pgfscope}%
\begin{pgfscope}%
\pgfsetbuttcap%
\pgfsetroundjoin%
\definecolor{currentfill}{rgb}{0.000000,0.000000,0.000000}%
\pgfsetfillcolor{currentfill}%
\pgfsetlinewidth{0.803000pt}%
\definecolor{currentstroke}{rgb}{0.000000,0.000000,0.000000}%
\pgfsetstrokecolor{currentstroke}%
\pgfsetdash{}{0pt}%
\pgfsys@defobject{currentmarker}{\pgfqpoint{0.000000in}{-0.048611in}}{\pgfqpoint{0.000000in}{0.000000in}}{%
\pgfpathmoveto{\pgfqpoint{0.000000in}{0.000000in}}%
\pgfpathlineto{\pgfqpoint{0.000000in}{-0.048611in}}%
\pgfusepath{stroke,fill}%
}%
\begin{pgfscope}%
\pgfsys@transformshift{0.854998in}{2.423071in}%
\pgfsys@useobject{currentmarker}{}%
\end{pgfscope}%
\end{pgfscope}%
\begin{pgfscope}%
\pgfpathrectangle{\pgfqpoint{0.706528in}{2.423071in}}{\pgfqpoint{3.266354in}{1.486960in}} %
\pgfusepath{clip}%
\pgfsetrectcap%
\pgfsetroundjoin%
\pgfsetlinewidth{0.803000pt}%
\definecolor{currentstroke}{rgb}{0.690196,0.690196,0.690196}%
\pgfsetstrokecolor{currentstroke}%
\pgfsetdash{}{0pt}%
\pgfpathmoveto{\pgfqpoint{1.448881in}{2.423071in}}%
\pgfpathlineto{\pgfqpoint{1.448881in}{3.910031in}}%
\pgfusepath{stroke}%
\end{pgfscope}%
\begin{pgfscope}%
\pgfsetbuttcap%
\pgfsetroundjoin%
\definecolor{currentfill}{rgb}{0.000000,0.000000,0.000000}%
\pgfsetfillcolor{currentfill}%
\pgfsetlinewidth{0.803000pt}%
\definecolor{currentstroke}{rgb}{0.000000,0.000000,0.000000}%
\pgfsetstrokecolor{currentstroke}%
\pgfsetdash{}{0pt}%
\pgfsys@defobject{currentmarker}{\pgfqpoint{0.000000in}{-0.048611in}}{\pgfqpoint{0.000000in}{0.000000in}}{%
\pgfpathmoveto{\pgfqpoint{0.000000in}{0.000000in}}%
\pgfpathlineto{\pgfqpoint{0.000000in}{-0.048611in}}%
\pgfusepath{stroke,fill}%
}%
\begin{pgfscope}%
\pgfsys@transformshift{1.448881in}{2.423071in}%
\pgfsys@useobject{currentmarker}{}%
\end{pgfscope}%
\end{pgfscope}%
\begin{pgfscope}%
\pgfpathrectangle{\pgfqpoint{0.706528in}{2.423071in}}{\pgfqpoint{3.266354in}{1.486960in}} %
\pgfusepath{clip}%
\pgfsetrectcap%
\pgfsetroundjoin%
\pgfsetlinewidth{0.803000pt}%
\definecolor{currentstroke}{rgb}{0.690196,0.690196,0.690196}%
\pgfsetstrokecolor{currentstroke}%
\pgfsetdash{}{0pt}%
\pgfpathmoveto{\pgfqpoint{2.042763in}{2.423071in}}%
\pgfpathlineto{\pgfqpoint{2.042763in}{3.910031in}}%
\pgfusepath{stroke}%
\end{pgfscope}%
\begin{pgfscope}%
\pgfsetbuttcap%
\pgfsetroundjoin%
\definecolor{currentfill}{rgb}{0.000000,0.000000,0.000000}%
\pgfsetfillcolor{currentfill}%
\pgfsetlinewidth{0.803000pt}%
\definecolor{currentstroke}{rgb}{0.000000,0.000000,0.000000}%
\pgfsetstrokecolor{currentstroke}%
\pgfsetdash{}{0pt}%
\pgfsys@defobject{currentmarker}{\pgfqpoint{0.000000in}{-0.048611in}}{\pgfqpoint{0.000000in}{0.000000in}}{%
\pgfpathmoveto{\pgfqpoint{0.000000in}{0.000000in}}%
\pgfpathlineto{\pgfqpoint{0.000000in}{-0.048611in}}%
\pgfusepath{stroke,fill}%
}%
\begin{pgfscope}%
\pgfsys@transformshift{2.042763in}{2.423071in}%
\pgfsys@useobject{currentmarker}{}%
\end{pgfscope}%
\end{pgfscope}%
\begin{pgfscope}%
\pgfpathrectangle{\pgfqpoint{0.706528in}{2.423071in}}{\pgfqpoint{3.266354in}{1.486960in}} %
\pgfusepath{clip}%
\pgfsetrectcap%
\pgfsetroundjoin%
\pgfsetlinewidth{0.803000pt}%
\definecolor{currentstroke}{rgb}{0.690196,0.690196,0.690196}%
\pgfsetstrokecolor{currentstroke}%
\pgfsetdash{}{0pt}%
\pgfpathmoveto{\pgfqpoint{2.636646in}{2.423071in}}%
\pgfpathlineto{\pgfqpoint{2.636646in}{3.910031in}}%
\pgfusepath{stroke}%
\end{pgfscope}%
\begin{pgfscope}%
\pgfsetbuttcap%
\pgfsetroundjoin%
\definecolor{currentfill}{rgb}{0.000000,0.000000,0.000000}%
\pgfsetfillcolor{currentfill}%
\pgfsetlinewidth{0.803000pt}%
\definecolor{currentstroke}{rgb}{0.000000,0.000000,0.000000}%
\pgfsetstrokecolor{currentstroke}%
\pgfsetdash{}{0pt}%
\pgfsys@defobject{currentmarker}{\pgfqpoint{0.000000in}{-0.048611in}}{\pgfqpoint{0.000000in}{0.000000in}}{%
\pgfpathmoveto{\pgfqpoint{0.000000in}{0.000000in}}%
\pgfpathlineto{\pgfqpoint{0.000000in}{-0.048611in}}%
\pgfusepath{stroke,fill}%
}%
\begin{pgfscope}%
\pgfsys@transformshift{2.636646in}{2.423071in}%
\pgfsys@useobject{currentmarker}{}%
\end{pgfscope}%
\end{pgfscope}%
\begin{pgfscope}%
\pgfpathrectangle{\pgfqpoint{0.706528in}{2.423071in}}{\pgfqpoint{3.266354in}{1.486960in}} %
\pgfusepath{clip}%
\pgfsetrectcap%
\pgfsetroundjoin%
\pgfsetlinewidth{0.803000pt}%
\definecolor{currentstroke}{rgb}{0.690196,0.690196,0.690196}%
\pgfsetstrokecolor{currentstroke}%
\pgfsetdash{}{0pt}%
\pgfpathmoveto{\pgfqpoint{3.230529in}{2.423071in}}%
\pgfpathlineto{\pgfqpoint{3.230529in}{3.910031in}}%
\pgfusepath{stroke}%
\end{pgfscope}%
\begin{pgfscope}%
\pgfsetbuttcap%
\pgfsetroundjoin%
\definecolor{currentfill}{rgb}{0.000000,0.000000,0.000000}%
\pgfsetfillcolor{currentfill}%
\pgfsetlinewidth{0.803000pt}%
\definecolor{currentstroke}{rgb}{0.000000,0.000000,0.000000}%
\pgfsetstrokecolor{currentstroke}%
\pgfsetdash{}{0pt}%
\pgfsys@defobject{currentmarker}{\pgfqpoint{0.000000in}{-0.048611in}}{\pgfqpoint{0.000000in}{0.000000in}}{%
\pgfpathmoveto{\pgfqpoint{0.000000in}{0.000000in}}%
\pgfpathlineto{\pgfqpoint{0.000000in}{-0.048611in}}%
\pgfusepath{stroke,fill}%
}%
\begin{pgfscope}%
\pgfsys@transformshift{3.230529in}{2.423071in}%
\pgfsys@useobject{currentmarker}{}%
\end{pgfscope}%
\end{pgfscope}%
\begin{pgfscope}%
\pgfpathrectangle{\pgfqpoint{0.706528in}{2.423071in}}{\pgfqpoint{3.266354in}{1.486960in}} %
\pgfusepath{clip}%
\pgfsetrectcap%
\pgfsetroundjoin%
\pgfsetlinewidth{0.803000pt}%
\definecolor{currentstroke}{rgb}{0.690196,0.690196,0.690196}%
\pgfsetstrokecolor{currentstroke}%
\pgfsetdash{}{0pt}%
\pgfpathmoveto{\pgfqpoint{3.824411in}{2.423071in}}%
\pgfpathlineto{\pgfqpoint{3.824411in}{3.910031in}}%
\pgfusepath{stroke}%
\end{pgfscope}%
\begin{pgfscope}%
\pgfsetbuttcap%
\pgfsetroundjoin%
\definecolor{currentfill}{rgb}{0.000000,0.000000,0.000000}%
\pgfsetfillcolor{currentfill}%
\pgfsetlinewidth{0.803000pt}%
\definecolor{currentstroke}{rgb}{0.000000,0.000000,0.000000}%
\pgfsetstrokecolor{currentstroke}%
\pgfsetdash{}{0pt}%
\pgfsys@defobject{currentmarker}{\pgfqpoint{0.000000in}{-0.048611in}}{\pgfqpoint{0.000000in}{0.000000in}}{%
\pgfpathmoveto{\pgfqpoint{0.000000in}{0.000000in}}%
\pgfpathlineto{\pgfqpoint{0.000000in}{-0.048611in}}%
\pgfusepath{stroke,fill}%
}%
\begin{pgfscope}%
\pgfsys@transformshift{3.824411in}{2.423071in}%
\pgfsys@useobject{currentmarker}{}%
\end{pgfscope}%
\end{pgfscope}%
\begin{pgfscope}%
\pgftext[x=2.339705in,y=2.367515in,,top]{\sffamily\fontsize{10.000000}{12.000000}\selectfont t in s}%
\end{pgfscope}%
\begin{pgfscope}%
\pgfpathrectangle{\pgfqpoint{0.706528in}{2.423071in}}{\pgfqpoint{3.266354in}{1.486960in}} %
\pgfusepath{clip}%
\pgfsetrectcap%
\pgfsetroundjoin%
\pgfsetlinewidth{0.803000pt}%
\definecolor{currentstroke}{rgb}{0.690196,0.690196,0.690196}%
\pgfsetstrokecolor{currentstroke}%
\pgfsetdash{}{0pt}%
\pgfpathmoveto{\pgfqpoint{0.706528in}{2.490660in}}%
\pgfpathlineto{\pgfqpoint{3.972882in}{2.490660in}}%
\pgfusepath{stroke}%
\end{pgfscope}%
\begin{pgfscope}%
\pgfsetbuttcap%
\pgfsetroundjoin%
\definecolor{currentfill}{rgb}{0.000000,0.000000,0.000000}%
\pgfsetfillcolor{currentfill}%
\pgfsetlinewidth{0.803000pt}%
\definecolor{currentstroke}{rgb}{0.000000,0.000000,0.000000}%
\pgfsetstrokecolor{currentstroke}%
\pgfsetdash{}{0pt}%
\pgfsys@defobject{currentmarker}{\pgfqpoint{-0.048611in}{0.000000in}}{\pgfqpoint{0.000000in}{0.000000in}}{%
\pgfpathmoveto{\pgfqpoint{0.000000in}{0.000000in}}%
\pgfpathlineto{\pgfqpoint{-0.048611in}{0.000000in}}%
\pgfusepath{stroke,fill}%
}%
\begin{pgfscope}%
\pgfsys@transformshift{0.706528in}{2.490660in}%
\pgfsys@useobject{currentmarker}{}%
\end{pgfscope}%
\end{pgfscope}%
\begin{pgfscope}%
\pgftext[x=0.529824in,y=2.437898in,left,base]{\sffamily\fontsize{10.000000}{12.000000}\selectfont 0}%
\end{pgfscope}%
\begin{pgfscope}%
\pgfpathrectangle{\pgfqpoint{0.706528in}{2.423071in}}{\pgfqpoint{3.266354in}{1.486960in}} %
\pgfusepath{clip}%
\pgfsetrectcap%
\pgfsetroundjoin%
\pgfsetlinewidth{0.803000pt}%
\definecolor{currentstroke}{rgb}{0.690196,0.690196,0.690196}%
\pgfsetstrokecolor{currentstroke}%
\pgfsetdash{}{0pt}%
\pgfpathmoveto{\pgfqpoint{0.706528in}{2.823292in}}%
\pgfpathlineto{\pgfqpoint{3.972882in}{2.823292in}}%
\pgfusepath{stroke}%
\end{pgfscope}%
\begin{pgfscope}%
\pgfsetbuttcap%
\pgfsetroundjoin%
\definecolor{currentfill}{rgb}{0.000000,0.000000,0.000000}%
\pgfsetfillcolor{currentfill}%
\pgfsetlinewidth{0.803000pt}%
\definecolor{currentstroke}{rgb}{0.000000,0.000000,0.000000}%
\pgfsetstrokecolor{currentstroke}%
\pgfsetdash{}{0pt}%
\pgfsys@defobject{currentmarker}{\pgfqpoint{-0.048611in}{0.000000in}}{\pgfqpoint{0.000000in}{0.000000in}}{%
\pgfpathmoveto{\pgfqpoint{0.000000in}{0.000000in}}%
\pgfpathlineto{\pgfqpoint{-0.048611in}{0.000000in}}%
\pgfusepath{stroke,fill}%
}%
\begin{pgfscope}%
\pgfsys@transformshift{0.706528in}{2.823292in}%
\pgfsys@useobject{currentmarker}{}%
\end{pgfscope}%
\end{pgfscope}%
\begin{pgfscope}%
\pgftext[x=0.450343in,y=2.770531in,left,base]{\sffamily\fontsize{10.000000}{12.000000}\selectfont 25}%
\end{pgfscope}%
\begin{pgfscope}%
\pgfpathrectangle{\pgfqpoint{0.706528in}{2.423071in}}{\pgfqpoint{3.266354in}{1.486960in}} %
\pgfusepath{clip}%
\pgfsetrectcap%
\pgfsetroundjoin%
\pgfsetlinewidth{0.803000pt}%
\definecolor{currentstroke}{rgb}{0.690196,0.690196,0.690196}%
\pgfsetstrokecolor{currentstroke}%
\pgfsetdash{}{0pt}%
\pgfpathmoveto{\pgfqpoint{0.706528in}{3.155924in}}%
\pgfpathlineto{\pgfqpoint{3.972882in}{3.155924in}}%
\pgfusepath{stroke}%
\end{pgfscope}%
\begin{pgfscope}%
\pgfsetbuttcap%
\pgfsetroundjoin%
\definecolor{currentfill}{rgb}{0.000000,0.000000,0.000000}%
\pgfsetfillcolor{currentfill}%
\pgfsetlinewidth{0.803000pt}%
\definecolor{currentstroke}{rgb}{0.000000,0.000000,0.000000}%
\pgfsetstrokecolor{currentstroke}%
\pgfsetdash{}{0pt}%
\pgfsys@defobject{currentmarker}{\pgfqpoint{-0.048611in}{0.000000in}}{\pgfqpoint{0.000000in}{0.000000in}}{%
\pgfpathmoveto{\pgfqpoint{0.000000in}{0.000000in}}%
\pgfpathlineto{\pgfqpoint{-0.048611in}{0.000000in}}%
\pgfusepath{stroke,fill}%
}%
\begin{pgfscope}%
\pgfsys@transformshift{0.706528in}{3.155924in}%
\pgfsys@useobject{currentmarker}{}%
\end{pgfscope}%
\end{pgfscope}%
\begin{pgfscope}%
\pgftext[x=0.450343in,y=3.103163in,left,base]{\sffamily\fontsize{10.000000}{12.000000}\selectfont 50}%
\end{pgfscope}%
\begin{pgfscope}%
\pgfpathrectangle{\pgfqpoint{0.706528in}{2.423071in}}{\pgfqpoint{3.266354in}{1.486960in}} %
\pgfusepath{clip}%
\pgfsetrectcap%
\pgfsetroundjoin%
\pgfsetlinewidth{0.803000pt}%
\definecolor{currentstroke}{rgb}{0.690196,0.690196,0.690196}%
\pgfsetstrokecolor{currentstroke}%
\pgfsetdash{}{0pt}%
\pgfpathmoveto{\pgfqpoint{0.706528in}{3.488556in}}%
\pgfpathlineto{\pgfqpoint{3.972882in}{3.488556in}}%
\pgfusepath{stroke}%
\end{pgfscope}%
\begin{pgfscope}%
\pgfsetbuttcap%
\pgfsetroundjoin%
\definecolor{currentfill}{rgb}{0.000000,0.000000,0.000000}%
\pgfsetfillcolor{currentfill}%
\pgfsetlinewidth{0.803000pt}%
\definecolor{currentstroke}{rgb}{0.000000,0.000000,0.000000}%
\pgfsetstrokecolor{currentstroke}%
\pgfsetdash{}{0pt}%
\pgfsys@defobject{currentmarker}{\pgfqpoint{-0.048611in}{0.000000in}}{\pgfqpoint{0.000000in}{0.000000in}}{%
\pgfpathmoveto{\pgfqpoint{0.000000in}{0.000000in}}%
\pgfpathlineto{\pgfqpoint{-0.048611in}{0.000000in}}%
\pgfusepath{stroke,fill}%
}%
\begin{pgfscope}%
\pgfsys@transformshift{0.706528in}{3.488556in}%
\pgfsys@useobject{currentmarker}{}%
\end{pgfscope}%
\end{pgfscope}%
\begin{pgfscope}%
\pgftext[x=0.450343in,y=3.435795in,left,base]{\sffamily\fontsize{10.000000}{12.000000}\selectfont 75}%
\end{pgfscope}%
\begin{pgfscope}%
\pgfpathrectangle{\pgfqpoint{0.706528in}{2.423071in}}{\pgfqpoint{3.266354in}{1.486960in}} %
\pgfusepath{clip}%
\pgfsetrectcap%
\pgfsetroundjoin%
\pgfsetlinewidth{0.803000pt}%
\definecolor{currentstroke}{rgb}{0.690196,0.690196,0.690196}%
\pgfsetstrokecolor{currentstroke}%
\pgfsetdash{}{0pt}%
\pgfpathmoveto{\pgfqpoint{0.706528in}{3.821188in}}%
\pgfpathlineto{\pgfqpoint{3.972882in}{3.821188in}}%
\pgfusepath{stroke}%
\end{pgfscope}%
\begin{pgfscope}%
\pgfsetbuttcap%
\pgfsetroundjoin%
\definecolor{currentfill}{rgb}{0.000000,0.000000,0.000000}%
\pgfsetfillcolor{currentfill}%
\pgfsetlinewidth{0.803000pt}%
\definecolor{currentstroke}{rgb}{0.000000,0.000000,0.000000}%
\pgfsetstrokecolor{currentstroke}%
\pgfsetdash{}{0pt}%
\pgfsys@defobject{currentmarker}{\pgfqpoint{-0.048611in}{0.000000in}}{\pgfqpoint{0.000000in}{0.000000in}}{%
\pgfpathmoveto{\pgfqpoint{0.000000in}{0.000000in}}%
\pgfpathlineto{\pgfqpoint{-0.048611in}{0.000000in}}%
\pgfusepath{stroke,fill}%
}%
\begin{pgfscope}%
\pgfsys@transformshift{0.706528in}{3.821188in}%
\pgfsys@useobject{currentmarker}{}%
\end{pgfscope}%
\end{pgfscope}%
\begin{pgfscope}%
\pgftext[x=0.370862in,y=3.768427in,left,base]{\sffamily\fontsize{10.000000}{12.000000}\selectfont 100}%
\end{pgfscope}%
\begin{pgfscope}%
\pgftext[x=0.315306in,y=3.166551in,,bottom,rotate=90.000000]{\sffamily\fontsize{10.000000}{12.000000}\selectfont deg}%
\end{pgfscope}%
\begin{pgfscope}%
\pgfpathrectangle{\pgfqpoint{0.706528in}{2.423071in}}{\pgfqpoint{3.266354in}{1.486960in}} %
\pgfusepath{clip}%
\pgfsetrectcap%
\pgfsetroundjoin%
\pgfsetlinewidth{1.003750pt}%
\definecolor{currentstroke}{rgb}{0.000000,0.500000,0.000000}%
\pgfsetstrokecolor{currentstroke}%
\pgfsetdash{}{0pt}%
\pgfpathmoveto{\pgfqpoint{0.854998in}{2.490660in}}%
\pgfpathlineto{\pgfqpoint{0.938142in}{2.579974in}}%
\pgfpathlineto{\pgfqpoint{1.021286in}{2.666235in}}%
\pgfpathlineto{\pgfqpoint{1.104429in}{2.749444in}}%
\pgfpathlineto{\pgfqpoint{1.187573in}{2.829601in}}%
\pgfpathlineto{\pgfqpoint{1.270716in}{2.906706in}}%
\pgfpathlineto{\pgfqpoint{1.353860in}{2.980758in}}%
\pgfpathlineto{\pgfqpoint{1.437003in}{3.051759in}}%
\pgfpathlineto{\pgfqpoint{1.520147in}{3.119707in}}%
\pgfpathlineto{\pgfqpoint{1.603290in}{3.184603in}}%
\pgfpathlineto{\pgfqpoint{1.686434in}{3.246446in}}%
\pgfpathlineto{\pgfqpoint{1.769577in}{3.305238in}}%
\pgfpathlineto{\pgfqpoint{1.852721in}{3.360977in}}%
\pgfpathlineto{\pgfqpoint{1.935865in}{3.413664in}}%
\pgfpathlineto{\pgfqpoint{2.007131in}{3.456395in}}%
\pgfpathlineto{\pgfqpoint{2.078396in}{3.496883in}}%
\pgfpathlineto{\pgfqpoint{2.149662in}{3.535130in}}%
\pgfpathlineto{\pgfqpoint{2.220928in}{3.571133in}}%
\pgfpathlineto{\pgfqpoint{2.292194in}{3.604894in}}%
\pgfpathlineto{\pgfqpoint{2.363460in}{3.636413in}}%
\pgfpathlineto{\pgfqpoint{2.434726in}{3.665690in}}%
\pgfpathlineto{\pgfqpoint{2.505992in}{3.692723in}}%
\pgfpathlineto{\pgfqpoint{2.577258in}{3.717515in}}%
\pgfpathlineto{\pgfqpoint{2.648524in}{3.740064in}}%
\pgfpathlineto{\pgfqpoint{2.719790in}{3.760370in}}%
\pgfpathlineto{\pgfqpoint{2.791055in}{3.778435in}}%
\pgfpathlineto{\pgfqpoint{2.862321in}{3.794256in}}%
\pgfpathlineto{\pgfqpoint{2.933587in}{3.807836in}}%
\pgfpathlineto{\pgfqpoint{3.004853in}{3.819172in}}%
\pgfpathlineto{\pgfqpoint{3.076119in}{3.828267in}}%
\pgfpathlineto{\pgfqpoint{3.147385in}{3.835119in}}%
\pgfpathlineto{\pgfqpoint{3.218651in}{3.839728in}}%
\pgfpathlineto{\pgfqpoint{3.289917in}{3.842095in}}%
\pgfpathlineto{\pgfqpoint{3.384938in}{3.842442in}}%
\pgfpathlineto{\pgfqpoint{3.824411in}{3.842442in}}%
\pgfpathlineto{\pgfqpoint{3.824411in}{3.842442in}}%
\pgfusepath{stroke}%
\end{pgfscope}%
\begin{pgfscope}%
\pgfsetrectcap%
\pgfsetmiterjoin%
\pgfsetlinewidth{0.803000pt}%
\definecolor{currentstroke}{rgb}{0.000000,0.000000,0.000000}%
\pgfsetstrokecolor{currentstroke}%
\pgfsetdash{}{0pt}%
\pgfpathmoveto{\pgfqpoint{0.706528in}{2.423071in}}%
\pgfpathlineto{\pgfqpoint{0.706528in}{3.910031in}}%
\pgfusepath{stroke}%
\end{pgfscope}%
\begin{pgfscope}%
\pgfsetrectcap%
\pgfsetmiterjoin%
\pgfsetlinewidth{0.803000pt}%
\definecolor{currentstroke}{rgb}{0.000000,0.000000,0.000000}%
\pgfsetstrokecolor{currentstroke}%
\pgfsetdash{}{0pt}%
\pgfpathmoveto{\pgfqpoint{3.972882in}{2.423071in}}%
\pgfpathlineto{\pgfqpoint{3.972882in}{3.910031in}}%
\pgfusepath{stroke}%
\end{pgfscope}%
\begin{pgfscope}%
\pgfsetrectcap%
\pgfsetmiterjoin%
\pgfsetlinewidth{0.803000pt}%
\definecolor{currentstroke}{rgb}{0.000000,0.000000,0.000000}%
\pgfsetstrokecolor{currentstroke}%
\pgfsetdash{}{0pt}%
\pgfpathmoveto{\pgfqpoint{0.706528in}{2.423071in}}%
\pgfpathlineto{\pgfqpoint{3.972882in}{2.423071in}}%
\pgfusepath{stroke}%
\end{pgfscope}%
\begin{pgfscope}%
\pgfsetrectcap%
\pgfsetmiterjoin%
\pgfsetlinewidth{0.803000pt}%
\definecolor{currentstroke}{rgb}{0.000000,0.000000,0.000000}%
\pgfsetstrokecolor{currentstroke}%
\pgfsetdash{}{0pt}%
\pgfpathmoveto{\pgfqpoint{0.706528in}{3.910031in}}%
\pgfpathlineto{\pgfqpoint{3.972882in}{3.910031in}}%
\pgfusepath{stroke}%
\end{pgfscope}%
\begin{pgfscope}%
\pgftext[x=2.339705in,y=3.993364in,,base]{\sffamily\fontsize{12.000000}{14.400000}\selectfont Velocity / steering angle}%
\end{pgfscope}%
\begin{pgfscope}%
\pgfsetbuttcap%
\pgfsetmiterjoin%
\definecolor{currentfill}{rgb}{1.000000,1.000000,1.000000}%
\pgfsetfillcolor{currentfill}%
\pgfsetfillopacity{0.800000}%
\pgfsetlinewidth{1.003750pt}%
\definecolor{currentstroke}{rgb}{0.800000,0.800000,0.800000}%
\pgfsetstrokecolor{currentstroke}%
\pgfsetstrokeopacity{0.800000}%
\pgfsetdash{}{0pt}%
\pgfpathmoveto{\pgfqpoint{0.803750in}{3.589230in}}%
\pgfpathlineto{\pgfqpoint{1.475433in}{3.589230in}}%
\pgfpathquadraticcurveto{\pgfqpoint{1.503210in}{3.589230in}}{\pgfqpoint{1.503210in}{3.617008in}}%
\pgfpathlineto{\pgfqpoint{1.503210in}{3.812808in}}%
\pgfpathquadraticcurveto{\pgfqpoint{1.503210in}{3.840586in}}{\pgfqpoint{1.475433in}{3.840586in}}%
\pgfpathlineto{\pgfqpoint{0.803750in}{3.840586in}}%
\pgfpathquadraticcurveto{\pgfqpoint{0.775972in}{3.840586in}}{\pgfqpoint{0.775972in}{3.812808in}}%
\pgfpathlineto{\pgfqpoint{0.775972in}{3.617008in}}%
\pgfpathquadraticcurveto{\pgfqpoint{0.775972in}{3.589230in}}{\pgfqpoint{0.803750in}{3.589230in}}%
\pgfpathclose%
\pgfusepath{stroke,fill}%
\end{pgfscope}%
\begin{pgfscope}%
\pgfsetrectcap%
\pgfsetroundjoin%
\pgfsetlinewidth{1.003750pt}%
\definecolor{currentstroke}{rgb}{0.000000,0.500000,0.000000}%
\pgfsetstrokecolor{currentstroke}%
\pgfsetdash{}{0pt}%
\pgfpathmoveto{\pgfqpoint{0.831528in}{3.728119in}}%
\pgfpathlineto{\pgfqpoint{1.109306in}{3.728119in}}%
\pgfusepath{stroke}%
\end{pgfscope}%
\begin{pgfscope}%
\pgftext[x=1.220417in,y=3.679508in,left,base]{\sffamily\fontsize{10.000000}{12.000000}\selectfont \(\displaystyle \theta(t)\)}%
\end{pgfscope}%
\begin{pgfscope}%
\pgfsetbuttcap%
\pgfsetmiterjoin%
\definecolor{currentfill}{rgb}{1.000000,1.000000,1.000000}%
\pgfsetfillcolor{currentfill}%
\pgfsetlinewidth{0.000000pt}%
\definecolor{currentstroke}{rgb}{0.000000,0.000000,0.000000}%
\pgfsetstrokecolor{currentstroke}%
\pgfsetstrokeopacity{0.000000}%
\pgfsetdash{}{0pt}%
\pgfpathmoveto{\pgfqpoint{0.706528in}{0.387222in}}%
\pgfpathlineto{\pgfqpoint{3.972882in}{0.387222in}}%
\pgfpathlineto{\pgfqpoint{3.972882in}{1.874182in}}%
\pgfpathlineto{\pgfqpoint{0.706528in}{1.874182in}}%
\pgfpathclose%
\pgfusepath{fill}%
\end{pgfscope}%
\begin{pgfscope}%
\pgfpathrectangle{\pgfqpoint{0.706528in}{0.387222in}}{\pgfqpoint{3.266354in}{1.486960in}} %
\pgfusepath{clip}%
\pgfsetrectcap%
\pgfsetroundjoin%
\pgfsetlinewidth{0.803000pt}%
\definecolor{currentstroke}{rgb}{0.690196,0.690196,0.690196}%
\pgfsetstrokecolor{currentstroke}%
\pgfsetdash{}{0pt}%
\pgfpathmoveto{\pgfqpoint{0.854998in}{0.387222in}}%
\pgfpathlineto{\pgfqpoint{0.854998in}{1.874182in}}%
\pgfusepath{stroke}%
\end{pgfscope}%
\begin{pgfscope}%
\pgfsetbuttcap%
\pgfsetroundjoin%
\definecolor{currentfill}{rgb}{0.000000,0.000000,0.000000}%
\pgfsetfillcolor{currentfill}%
\pgfsetlinewidth{0.803000pt}%
\definecolor{currentstroke}{rgb}{0.000000,0.000000,0.000000}%
\pgfsetstrokecolor{currentstroke}%
\pgfsetdash{}{0pt}%
\pgfsys@defobject{currentmarker}{\pgfqpoint{0.000000in}{-0.048611in}}{\pgfqpoint{0.000000in}{0.000000in}}{%
\pgfpathmoveto{\pgfqpoint{0.000000in}{0.000000in}}%
\pgfpathlineto{\pgfqpoint{0.000000in}{-0.048611in}}%
\pgfusepath{stroke,fill}%
}%
\begin{pgfscope}%
\pgfsys@transformshift{0.854998in}{0.387222in}%
\pgfsys@useobject{currentmarker}{}%
\end{pgfscope}%
\end{pgfscope}%
\begin{pgfscope}%
\pgftext[x=0.854998in,y=0.290000in,,top]{\sffamily\fontsize{10.000000}{12.000000}\selectfont 0}%
\end{pgfscope}%
\begin{pgfscope}%
\pgfpathrectangle{\pgfqpoint{0.706528in}{0.387222in}}{\pgfqpoint{3.266354in}{1.486960in}} %
\pgfusepath{clip}%
\pgfsetrectcap%
\pgfsetroundjoin%
\pgfsetlinewidth{0.803000pt}%
\definecolor{currentstroke}{rgb}{0.690196,0.690196,0.690196}%
\pgfsetstrokecolor{currentstroke}%
\pgfsetdash{}{0pt}%
\pgfpathmoveto{\pgfqpoint{1.448881in}{0.387222in}}%
\pgfpathlineto{\pgfqpoint{1.448881in}{1.874182in}}%
\pgfusepath{stroke}%
\end{pgfscope}%
\begin{pgfscope}%
\pgfsetbuttcap%
\pgfsetroundjoin%
\definecolor{currentfill}{rgb}{0.000000,0.000000,0.000000}%
\pgfsetfillcolor{currentfill}%
\pgfsetlinewidth{0.803000pt}%
\definecolor{currentstroke}{rgb}{0.000000,0.000000,0.000000}%
\pgfsetstrokecolor{currentstroke}%
\pgfsetdash{}{0pt}%
\pgfsys@defobject{currentmarker}{\pgfqpoint{0.000000in}{-0.048611in}}{\pgfqpoint{0.000000in}{0.000000in}}{%
\pgfpathmoveto{\pgfqpoint{0.000000in}{0.000000in}}%
\pgfpathlineto{\pgfqpoint{0.000000in}{-0.048611in}}%
\pgfusepath{stroke,fill}%
}%
\begin{pgfscope}%
\pgfsys@transformshift{1.448881in}{0.387222in}%
\pgfsys@useobject{currentmarker}{}%
\end{pgfscope}%
\end{pgfscope}%
\begin{pgfscope}%
\pgftext[x=1.448881in,y=0.290000in,,top]{\sffamily\fontsize{10.000000}{12.000000}\selectfont 2}%
\end{pgfscope}%
\begin{pgfscope}%
\pgfpathrectangle{\pgfqpoint{0.706528in}{0.387222in}}{\pgfqpoint{3.266354in}{1.486960in}} %
\pgfusepath{clip}%
\pgfsetrectcap%
\pgfsetroundjoin%
\pgfsetlinewidth{0.803000pt}%
\definecolor{currentstroke}{rgb}{0.690196,0.690196,0.690196}%
\pgfsetstrokecolor{currentstroke}%
\pgfsetdash{}{0pt}%
\pgfpathmoveto{\pgfqpoint{2.042763in}{0.387222in}}%
\pgfpathlineto{\pgfqpoint{2.042763in}{1.874182in}}%
\pgfusepath{stroke}%
\end{pgfscope}%
\begin{pgfscope}%
\pgfsetbuttcap%
\pgfsetroundjoin%
\definecolor{currentfill}{rgb}{0.000000,0.000000,0.000000}%
\pgfsetfillcolor{currentfill}%
\pgfsetlinewidth{0.803000pt}%
\definecolor{currentstroke}{rgb}{0.000000,0.000000,0.000000}%
\pgfsetstrokecolor{currentstroke}%
\pgfsetdash{}{0pt}%
\pgfsys@defobject{currentmarker}{\pgfqpoint{0.000000in}{-0.048611in}}{\pgfqpoint{0.000000in}{0.000000in}}{%
\pgfpathmoveto{\pgfqpoint{0.000000in}{0.000000in}}%
\pgfpathlineto{\pgfqpoint{0.000000in}{-0.048611in}}%
\pgfusepath{stroke,fill}%
}%
\begin{pgfscope}%
\pgfsys@transformshift{2.042763in}{0.387222in}%
\pgfsys@useobject{currentmarker}{}%
\end{pgfscope}%
\end{pgfscope}%
\begin{pgfscope}%
\pgftext[x=2.042763in,y=0.290000in,,top]{\sffamily\fontsize{10.000000}{12.000000}\selectfont 4}%
\end{pgfscope}%
\begin{pgfscope}%
\pgfpathrectangle{\pgfqpoint{0.706528in}{0.387222in}}{\pgfqpoint{3.266354in}{1.486960in}} %
\pgfusepath{clip}%
\pgfsetrectcap%
\pgfsetroundjoin%
\pgfsetlinewidth{0.803000pt}%
\definecolor{currentstroke}{rgb}{0.690196,0.690196,0.690196}%
\pgfsetstrokecolor{currentstroke}%
\pgfsetdash{}{0pt}%
\pgfpathmoveto{\pgfqpoint{2.636646in}{0.387222in}}%
\pgfpathlineto{\pgfqpoint{2.636646in}{1.874182in}}%
\pgfusepath{stroke}%
\end{pgfscope}%
\begin{pgfscope}%
\pgfsetbuttcap%
\pgfsetroundjoin%
\definecolor{currentfill}{rgb}{0.000000,0.000000,0.000000}%
\pgfsetfillcolor{currentfill}%
\pgfsetlinewidth{0.803000pt}%
\definecolor{currentstroke}{rgb}{0.000000,0.000000,0.000000}%
\pgfsetstrokecolor{currentstroke}%
\pgfsetdash{}{0pt}%
\pgfsys@defobject{currentmarker}{\pgfqpoint{0.000000in}{-0.048611in}}{\pgfqpoint{0.000000in}{0.000000in}}{%
\pgfpathmoveto{\pgfqpoint{0.000000in}{0.000000in}}%
\pgfpathlineto{\pgfqpoint{0.000000in}{-0.048611in}}%
\pgfusepath{stroke,fill}%
}%
\begin{pgfscope}%
\pgfsys@transformshift{2.636646in}{0.387222in}%
\pgfsys@useobject{currentmarker}{}%
\end{pgfscope}%
\end{pgfscope}%
\begin{pgfscope}%
\pgftext[x=2.636646in,y=0.290000in,,top]{\sffamily\fontsize{10.000000}{12.000000}\selectfont 6}%
\end{pgfscope}%
\begin{pgfscope}%
\pgfpathrectangle{\pgfqpoint{0.706528in}{0.387222in}}{\pgfqpoint{3.266354in}{1.486960in}} %
\pgfusepath{clip}%
\pgfsetrectcap%
\pgfsetroundjoin%
\pgfsetlinewidth{0.803000pt}%
\definecolor{currentstroke}{rgb}{0.690196,0.690196,0.690196}%
\pgfsetstrokecolor{currentstroke}%
\pgfsetdash{}{0pt}%
\pgfpathmoveto{\pgfqpoint{3.230529in}{0.387222in}}%
\pgfpathlineto{\pgfqpoint{3.230529in}{1.874182in}}%
\pgfusepath{stroke}%
\end{pgfscope}%
\begin{pgfscope}%
\pgfsetbuttcap%
\pgfsetroundjoin%
\definecolor{currentfill}{rgb}{0.000000,0.000000,0.000000}%
\pgfsetfillcolor{currentfill}%
\pgfsetlinewidth{0.803000pt}%
\definecolor{currentstroke}{rgb}{0.000000,0.000000,0.000000}%
\pgfsetstrokecolor{currentstroke}%
\pgfsetdash{}{0pt}%
\pgfsys@defobject{currentmarker}{\pgfqpoint{0.000000in}{-0.048611in}}{\pgfqpoint{0.000000in}{0.000000in}}{%
\pgfpathmoveto{\pgfqpoint{0.000000in}{0.000000in}}%
\pgfpathlineto{\pgfqpoint{0.000000in}{-0.048611in}}%
\pgfusepath{stroke,fill}%
}%
\begin{pgfscope}%
\pgfsys@transformshift{3.230529in}{0.387222in}%
\pgfsys@useobject{currentmarker}{}%
\end{pgfscope}%
\end{pgfscope}%
\begin{pgfscope}%
\pgftext[x=3.230529in,y=0.290000in,,top]{\sffamily\fontsize{10.000000}{12.000000}\selectfont 8}%
\end{pgfscope}%
\begin{pgfscope}%
\pgfpathrectangle{\pgfqpoint{0.706528in}{0.387222in}}{\pgfqpoint{3.266354in}{1.486960in}} %
\pgfusepath{clip}%
\pgfsetrectcap%
\pgfsetroundjoin%
\pgfsetlinewidth{0.803000pt}%
\definecolor{currentstroke}{rgb}{0.690196,0.690196,0.690196}%
\pgfsetstrokecolor{currentstroke}%
\pgfsetdash{}{0pt}%
\pgfpathmoveto{\pgfqpoint{3.824411in}{0.387222in}}%
\pgfpathlineto{\pgfqpoint{3.824411in}{1.874182in}}%
\pgfusepath{stroke}%
\end{pgfscope}%
\begin{pgfscope}%
\pgfsetbuttcap%
\pgfsetroundjoin%
\definecolor{currentfill}{rgb}{0.000000,0.000000,0.000000}%
\pgfsetfillcolor{currentfill}%
\pgfsetlinewidth{0.803000pt}%
\definecolor{currentstroke}{rgb}{0.000000,0.000000,0.000000}%
\pgfsetstrokecolor{currentstroke}%
\pgfsetdash{}{0pt}%
\pgfsys@defobject{currentmarker}{\pgfqpoint{0.000000in}{-0.048611in}}{\pgfqpoint{0.000000in}{0.000000in}}{%
\pgfpathmoveto{\pgfqpoint{0.000000in}{0.000000in}}%
\pgfpathlineto{\pgfqpoint{0.000000in}{-0.048611in}}%
\pgfusepath{stroke,fill}%
}%
\begin{pgfscope}%
\pgfsys@transformshift{3.824411in}{0.387222in}%
\pgfsys@useobject{currentmarker}{}%
\end{pgfscope}%
\end{pgfscope}%
\begin{pgfscope}%
\pgftext[x=3.824411in,y=0.290000in,,top]{\sffamily\fontsize{10.000000}{12.000000}\selectfont 10}%
\end{pgfscope}%
\begin{pgfscope}%
\pgfpathrectangle{\pgfqpoint{0.706528in}{0.387222in}}{\pgfqpoint{3.266354in}{1.486960in}} %
\pgfusepath{clip}%
\pgfsetrectcap%
\pgfsetroundjoin%
\pgfsetlinewidth{0.803000pt}%
\definecolor{currentstroke}{rgb}{0.690196,0.690196,0.690196}%
\pgfsetstrokecolor{currentstroke}%
\pgfsetdash{}{0pt}%
\pgfpathmoveto{\pgfqpoint{0.706528in}{0.454811in}}%
\pgfpathlineto{\pgfqpoint{3.972882in}{0.454811in}}%
\pgfusepath{stroke}%
\end{pgfscope}%
\begin{pgfscope}%
\pgfsetbuttcap%
\pgfsetroundjoin%
\definecolor{currentfill}{rgb}{1.000000,0.000000,0.000000}%
\pgfsetfillcolor{currentfill}%
\pgfsetlinewidth{0.803000pt}%
\definecolor{currentstroke}{rgb}{1.000000,0.000000,0.000000}%
\pgfsetstrokecolor{currentstroke}%
\pgfsetdash{}{0pt}%
\pgfsys@defobject{currentmarker}{\pgfqpoint{-0.048611in}{0.000000in}}{\pgfqpoint{0.000000in}{0.000000in}}{%
\pgfpathmoveto{\pgfqpoint{0.000000in}{0.000000in}}%
\pgfpathlineto{\pgfqpoint{-0.048611in}{0.000000in}}%
\pgfusepath{stroke,fill}%
}%
\begin{pgfscope}%
\pgfsys@transformshift{0.706528in}{0.454811in}%
\pgfsys@useobject{currentmarker}{}%
\end{pgfscope}%
\end{pgfscope}%
\begin{pgfscope}%
\definecolor{textcolor}{rgb}{1.000000,0.000000,0.000000}%
\pgfsetstrokecolor{textcolor}%
\pgfsetfillcolor{textcolor}%
\pgftext[x=0.529824in,y=0.402050in,left,base]{\color{textcolor}\sffamily\fontsize{10.000000}{12.000000}\selectfont 0}%
\end{pgfscope}%
\begin{pgfscope}%
\pgfpathrectangle{\pgfqpoint{0.706528in}{0.387222in}}{\pgfqpoint{3.266354in}{1.486960in}} %
\pgfusepath{clip}%
\pgfsetrectcap%
\pgfsetroundjoin%
\pgfsetlinewidth{0.803000pt}%
\definecolor{currentstroke}{rgb}{0.690196,0.690196,0.690196}%
\pgfsetstrokecolor{currentstroke}%
\pgfsetdash{}{0pt}%
\pgfpathmoveto{\pgfqpoint{0.706528in}{0.926672in}}%
\pgfpathlineto{\pgfqpoint{3.972882in}{0.926672in}}%
\pgfusepath{stroke}%
\end{pgfscope}%
\begin{pgfscope}%
\pgfsetbuttcap%
\pgfsetroundjoin%
\definecolor{currentfill}{rgb}{1.000000,0.000000,0.000000}%
\pgfsetfillcolor{currentfill}%
\pgfsetlinewidth{0.803000pt}%
\definecolor{currentstroke}{rgb}{1.000000,0.000000,0.000000}%
\pgfsetstrokecolor{currentstroke}%
\pgfsetdash{}{0pt}%
\pgfsys@defobject{currentmarker}{\pgfqpoint{-0.048611in}{0.000000in}}{\pgfqpoint{0.000000in}{0.000000in}}{%
\pgfpathmoveto{\pgfqpoint{0.000000in}{0.000000in}}%
\pgfpathlineto{\pgfqpoint{-0.048611in}{0.000000in}}%
\pgfusepath{stroke,fill}%
}%
\begin{pgfscope}%
\pgfsys@transformshift{0.706528in}{0.926672in}%
\pgfsys@useobject{currentmarker}{}%
\end{pgfscope}%
\end{pgfscope}%
\begin{pgfscope}%
\definecolor{textcolor}{rgb}{1.000000,0.000000,0.000000}%
\pgfsetstrokecolor{textcolor}%
\pgfsetfillcolor{textcolor}%
\pgftext[x=0.450343in,y=0.873911in,left,base]{\color{textcolor}\sffamily\fontsize{10.000000}{12.000000}\selectfont 10}%
\end{pgfscope}%
\begin{pgfscope}%
\pgfpathrectangle{\pgfqpoint{0.706528in}{0.387222in}}{\pgfqpoint{3.266354in}{1.486960in}} %
\pgfusepath{clip}%
\pgfsetrectcap%
\pgfsetroundjoin%
\pgfsetlinewidth{0.803000pt}%
\definecolor{currentstroke}{rgb}{0.690196,0.690196,0.690196}%
\pgfsetstrokecolor{currentstroke}%
\pgfsetdash{}{0pt}%
\pgfpathmoveto{\pgfqpoint{0.706528in}{1.398533in}}%
\pgfpathlineto{\pgfqpoint{3.972882in}{1.398533in}}%
\pgfusepath{stroke}%
\end{pgfscope}%
\begin{pgfscope}%
\pgfsetbuttcap%
\pgfsetroundjoin%
\definecolor{currentfill}{rgb}{1.000000,0.000000,0.000000}%
\pgfsetfillcolor{currentfill}%
\pgfsetlinewidth{0.803000pt}%
\definecolor{currentstroke}{rgb}{1.000000,0.000000,0.000000}%
\pgfsetstrokecolor{currentstroke}%
\pgfsetdash{}{0pt}%
\pgfsys@defobject{currentmarker}{\pgfqpoint{-0.048611in}{0.000000in}}{\pgfqpoint{0.000000in}{0.000000in}}{%
\pgfpathmoveto{\pgfqpoint{0.000000in}{0.000000in}}%
\pgfpathlineto{\pgfqpoint{-0.048611in}{0.000000in}}%
\pgfusepath{stroke,fill}%
}%
\begin{pgfscope}%
\pgfsys@transformshift{0.706528in}{1.398533in}%
\pgfsys@useobject{currentmarker}{}%
\end{pgfscope}%
\end{pgfscope}%
\begin{pgfscope}%
\definecolor{textcolor}{rgb}{1.000000,0.000000,0.000000}%
\pgfsetstrokecolor{textcolor}%
\pgfsetfillcolor{textcolor}%
\pgftext[x=0.450343in,y=1.345771in,left,base]{\color{textcolor}\sffamily\fontsize{10.000000}{12.000000}\selectfont 20}%
\end{pgfscope}%
\begin{pgfscope}%
\pgfpathrectangle{\pgfqpoint{0.706528in}{0.387222in}}{\pgfqpoint{3.266354in}{1.486960in}} %
\pgfusepath{clip}%
\pgfsetrectcap%
\pgfsetroundjoin%
\pgfsetlinewidth{0.803000pt}%
\definecolor{currentstroke}{rgb}{0.690196,0.690196,0.690196}%
\pgfsetstrokecolor{currentstroke}%
\pgfsetdash{}{0pt}%
\pgfpathmoveto{\pgfqpoint{0.706528in}{1.870394in}}%
\pgfpathlineto{\pgfqpoint{3.972882in}{1.870394in}}%
\pgfusepath{stroke}%
\end{pgfscope}%
\begin{pgfscope}%
\pgfsetbuttcap%
\pgfsetroundjoin%
\definecolor{currentfill}{rgb}{1.000000,0.000000,0.000000}%
\pgfsetfillcolor{currentfill}%
\pgfsetlinewidth{0.803000pt}%
\definecolor{currentstroke}{rgb}{1.000000,0.000000,0.000000}%
\pgfsetstrokecolor{currentstroke}%
\pgfsetdash{}{0pt}%
\pgfsys@defobject{currentmarker}{\pgfqpoint{-0.048611in}{0.000000in}}{\pgfqpoint{0.000000in}{0.000000in}}{%
\pgfpathmoveto{\pgfqpoint{0.000000in}{0.000000in}}%
\pgfpathlineto{\pgfqpoint{-0.048611in}{0.000000in}}%
\pgfusepath{stroke,fill}%
}%
\begin{pgfscope}%
\pgfsys@transformshift{0.706528in}{1.870394in}%
\pgfsys@useobject{currentmarker}{}%
\end{pgfscope}%
\end{pgfscope}%
\begin{pgfscope}%
\definecolor{textcolor}{rgb}{1.000000,0.000000,0.000000}%
\pgfsetstrokecolor{textcolor}%
\pgfsetfillcolor{textcolor}%
\pgftext[x=0.450343in,y=1.817632in,left,base]{\color{textcolor}\sffamily\fontsize{10.000000}{12.000000}\selectfont 30}%
\end{pgfscope}%
\begin{pgfscope}%
\pgftext[x=0.394787in,y=1.130702in,,bottom,rotate=90.000000]{\sffamily\fontsize{10.000000}{12.000000}\selectfont m/s}%
\end{pgfscope}%
\begin{pgfscope}%
\pgfpathrectangle{\pgfqpoint{0.706528in}{0.387222in}}{\pgfqpoint{3.266354in}{1.486960in}} %
\pgfusepath{clip}%
\pgfsetrectcap%
\pgfsetroundjoin%
\pgfsetlinewidth{1.003750pt}%
\definecolor{currentstroke}{rgb}{1.000000,0.000000,0.000000}%
\pgfsetstrokecolor{currentstroke}%
\pgfsetdash{}{0pt}%
\pgfpathmoveto{\pgfqpoint{0.854998in}{1.806593in}}%
\pgfpathlineto{\pgfqpoint{3.325550in}{0.456974in}}%
\pgfpathlineto{\pgfqpoint{3.337427in}{0.454811in}}%
\pgfpathlineto{\pgfqpoint{3.824411in}{0.454811in}}%
\pgfpathlineto{\pgfqpoint{3.824411in}{0.454811in}}%
\pgfusepath{stroke}%
\end{pgfscope}%
\begin{pgfscope}%
\pgfsetrectcap%
\pgfsetmiterjoin%
\pgfsetlinewidth{0.803000pt}%
\definecolor{currentstroke}{rgb}{0.000000,0.000000,0.000000}%
\pgfsetstrokecolor{currentstroke}%
\pgfsetdash{}{0pt}%
\pgfpathmoveto{\pgfqpoint{0.706528in}{0.387222in}}%
\pgfpathlineto{\pgfqpoint{0.706528in}{1.874182in}}%
\pgfusepath{stroke}%
\end{pgfscope}%
\begin{pgfscope}%
\pgfsetrectcap%
\pgfsetmiterjoin%
\pgfsetlinewidth{0.803000pt}%
\definecolor{currentstroke}{rgb}{0.000000,0.000000,0.000000}%
\pgfsetstrokecolor{currentstroke}%
\pgfsetdash{}{0pt}%
\pgfpathmoveto{\pgfqpoint{3.972882in}{0.387222in}}%
\pgfpathlineto{\pgfqpoint{3.972882in}{1.874182in}}%
\pgfusepath{stroke}%
\end{pgfscope}%
\begin{pgfscope}%
\pgfsetrectcap%
\pgfsetmiterjoin%
\pgfsetlinewidth{0.803000pt}%
\definecolor{currentstroke}{rgb}{0.000000,0.000000,0.000000}%
\pgfsetstrokecolor{currentstroke}%
\pgfsetdash{}{0pt}%
\pgfpathmoveto{\pgfqpoint{0.706528in}{0.387222in}}%
\pgfpathlineto{\pgfqpoint{3.972882in}{0.387222in}}%
\pgfusepath{stroke}%
\end{pgfscope}%
\begin{pgfscope}%
\pgfsetrectcap%
\pgfsetmiterjoin%
\pgfsetlinewidth{0.803000pt}%
\definecolor{currentstroke}{rgb}{0.000000,0.000000,0.000000}%
\pgfsetstrokecolor{currentstroke}%
\pgfsetdash{}{0pt}%
\pgfpathmoveto{\pgfqpoint{0.706528in}{1.874182in}}%
\pgfpathlineto{\pgfqpoint{3.972882in}{1.874182in}}%
\pgfusepath{stroke}%
\end{pgfscope}%
\begin{pgfscope}%
\pgfsetbuttcap%
\pgfsetmiterjoin%
\definecolor{currentfill}{rgb}{1.000000,1.000000,1.000000}%
\pgfsetfillcolor{currentfill}%
\pgfsetfillopacity{0.800000}%
\pgfsetlinewidth{1.003750pt}%
\definecolor{currentstroke}{rgb}{0.800000,0.800000,0.800000}%
\pgfsetstrokecolor{currentstroke}%
\pgfsetstrokeopacity{0.800000}%
\pgfsetdash{}{0pt}%
\pgfpathmoveto{\pgfqpoint{3.190281in}{1.343691in}}%
\pgfpathlineto{\pgfqpoint{3.875659in}{1.343691in}}%
\pgfpathquadraticcurveto{\pgfqpoint{3.903437in}{1.343691in}}{\pgfqpoint{3.903437in}{1.371469in}}%
\pgfpathlineto{\pgfqpoint{3.903437in}{1.776960in}}%
\pgfpathquadraticcurveto{\pgfqpoint{3.903437in}{1.804738in}}{\pgfqpoint{3.875659in}{1.804738in}}%
\pgfpathlineto{\pgfqpoint{3.190281in}{1.804738in}}%
\pgfpathquadraticcurveto{\pgfqpoint{3.162503in}{1.804738in}}{\pgfqpoint{3.162503in}{1.776960in}}%
\pgfpathlineto{\pgfqpoint{3.162503in}{1.371469in}}%
\pgfpathquadraticcurveto{\pgfqpoint{3.162503in}{1.343691in}}{\pgfqpoint{3.190281in}{1.343691in}}%
\pgfpathclose%
\pgfusepath{stroke,fill}%
\end{pgfscope}%
\begin{pgfscope}%
\pgfsetrectcap%
\pgfsetroundjoin%
\pgfsetlinewidth{1.003750pt}%
\definecolor{currentstroke}{rgb}{1.000000,0.000000,0.000000}%
\pgfsetstrokecolor{currentstroke}%
\pgfsetdash{}{0pt}%
\pgfpathmoveto{\pgfqpoint{3.218059in}{1.692270in}}%
\pgfpathlineto{\pgfqpoint{3.495836in}{1.692270in}}%
\pgfusepath{stroke}%
\end{pgfscope}%
\begin{pgfscope}%
\pgftext[x=3.606948in,y=1.643659in,left,base]{\sffamily\fontsize{10.000000}{12.000000}\selectfont \(\displaystyle v(t)\)}%
\end{pgfscope}%
\begin{pgfscope}%
\pgfsetrectcap%
\pgfsetroundjoin%
\pgfsetlinewidth{1.003750pt}%
\definecolor{currentstroke}{rgb}{0.000000,0.000000,1.000000}%
\pgfsetstrokecolor{currentstroke}%
\pgfsetdash{}{0pt}%
\pgfpathmoveto{\pgfqpoint{3.218059in}{1.482580in}}%
\pgfpathlineto{\pgfqpoint{3.495836in}{1.482580in}}%
\pgfusepath{stroke}%
\end{pgfscope}%
\begin{pgfscope}%
\pgftext[x=3.606948in,y=1.433969in,left,base]{\sffamily\fontsize{10.000000}{12.000000}\selectfont \(\displaystyle \phi(t)\)}%
\end{pgfscope}%
\begin{pgfscope}%
\pgfsetbuttcap%
\pgfsetroundjoin%
\definecolor{currentfill}{rgb}{0.000000,0.000000,1.000000}%
\pgfsetfillcolor{currentfill}%
\pgfsetlinewidth{0.803000pt}%
\definecolor{currentstroke}{rgb}{0.000000,0.000000,1.000000}%
\pgfsetstrokecolor{currentstroke}%
\pgfsetdash{}{0pt}%
\pgfsys@defobject{currentmarker}{\pgfqpoint{0.000000in}{0.000000in}}{\pgfqpoint{0.048611in}{0.000000in}}{%
\pgfpathmoveto{\pgfqpoint{0.000000in}{0.000000in}}%
\pgfpathlineto{\pgfqpoint{0.048611in}{0.000000in}}%
\pgfusepath{stroke,fill}%
}%
\begin{pgfscope}%
\pgfsys@transformshift{3.972882in}{0.824988in}%
\pgfsys@useobject{currentmarker}{}%
\end{pgfscope}%
\end{pgfscope}%
\begin{pgfscope}%
\definecolor{textcolor}{rgb}{0.000000,0.000000,1.000000}%
\pgfsetstrokecolor{textcolor}%
\pgfsetfillcolor{textcolor}%
\pgftext[x=4.070104in,y=0.772227in,left,base]{\color{textcolor}\sffamily\fontsize{10.000000}{12.000000}\selectfont 14.0}%
\end{pgfscope}%
\begin{pgfscope}%
\pgfsetbuttcap%
\pgfsetroundjoin%
\definecolor{currentfill}{rgb}{0.000000,0.000000,1.000000}%
\pgfsetfillcolor{currentfill}%
\pgfsetlinewidth{0.803000pt}%
\definecolor{currentstroke}{rgb}{0.000000,0.000000,1.000000}%
\pgfsetstrokecolor{currentstroke}%
\pgfsetdash{}{0pt}%
\pgfsys@defobject{currentmarker}{\pgfqpoint{0.000000in}{0.000000in}}{\pgfqpoint{0.048611in}{0.000000in}}{%
\pgfpathmoveto{\pgfqpoint{0.000000in}{0.000000in}}%
\pgfpathlineto{\pgfqpoint{0.048611in}{0.000000in}}%
\pgfusepath{stroke,fill}%
}%
\begin{pgfscope}%
\pgfsys@transformshift{3.972882in}{1.296849in}%
\pgfsys@useobject{currentmarker}{}%
\end{pgfscope}%
\end{pgfscope}%
\begin{pgfscope}%
\definecolor{textcolor}{rgb}{0.000000,0.000000,1.000000}%
\pgfsetstrokecolor{textcolor}%
\pgfsetfillcolor{textcolor}%
\pgftext[x=4.070104in,y=1.244088in,left,base]{\color{textcolor}\sffamily\fontsize{10.000000}{12.000000}\selectfont 14.5}%
\end{pgfscope}%
\begin{pgfscope}%
\pgfsetbuttcap%
\pgfsetroundjoin%
\definecolor{currentfill}{rgb}{0.000000,0.000000,1.000000}%
\pgfsetfillcolor{currentfill}%
\pgfsetlinewidth{0.803000pt}%
\definecolor{currentstroke}{rgb}{0.000000,0.000000,1.000000}%
\pgfsetstrokecolor{currentstroke}%
\pgfsetdash{}{0pt}%
\pgfsys@defobject{currentmarker}{\pgfqpoint{0.000000in}{0.000000in}}{\pgfqpoint{0.048611in}{0.000000in}}{%
\pgfpathmoveto{\pgfqpoint{0.000000in}{0.000000in}}%
\pgfpathlineto{\pgfqpoint{0.048611in}{0.000000in}}%
\pgfusepath{stroke,fill}%
}%
\begin{pgfscope}%
\pgfsys@transformshift{3.972882in}{1.768710in}%
\pgfsys@useobject{currentmarker}{}%
\end{pgfscope}%
\end{pgfscope}%
\begin{pgfscope}%
\definecolor{textcolor}{rgb}{0.000000,0.000000,1.000000}%
\pgfsetstrokecolor{textcolor}%
\pgfsetfillcolor{textcolor}%
\pgftext[x=4.070104in,y=1.715948in,left,base]{\color{textcolor}\sffamily\fontsize{10.000000}{12.000000}\selectfont 15.0}%
\end{pgfscope}%
\begin{pgfscope}%
\pgftext[x=4.403776in,y=1.130702in,,top,rotate=90.000000]{\sffamily\fontsize{10.000000}{12.000000}\selectfont deg}%
\end{pgfscope}%
\begin{pgfscope}%
\pgfpathrectangle{\pgfqpoint{0.706528in}{0.387222in}}{\pgfqpoint{3.266354in}{1.486960in}} %
\pgfusepath{clip}%
\pgfsetrectcap%
\pgfsetroundjoin%
\pgfsetlinewidth{1.003750pt}%
\definecolor{currentstroke}{rgb}{0.000000,0.000000,1.000000}%
\pgfsetstrokecolor{currentstroke}%
\pgfsetdash{}{0pt}%
\pgfpathmoveto{\pgfqpoint{0.854998in}{1.130702in}}%
\pgfpathlineto{\pgfqpoint{3.824411in}{1.130702in}}%
\pgfpathlineto{\pgfqpoint{3.824411in}{1.130702in}}%
\pgfusepath{stroke}%
\end{pgfscope}%
\begin{pgfscope}%
\pgfsetrectcap%
\pgfsetmiterjoin%
\pgfsetlinewidth{0.803000pt}%
\definecolor{currentstroke}{rgb}{1.000000,0.000000,0.000000}%
\pgfsetstrokecolor{currentstroke}%
\pgfsetdash{}{0pt}%
\pgfpathmoveto{\pgfqpoint{0.706528in}{0.387222in}}%
\pgfpathlineto{\pgfqpoint{0.706528in}{1.874182in}}%
\pgfusepath{stroke}%
\end{pgfscope}%
\begin{pgfscope}%
\pgfsetrectcap%
\pgfsetmiterjoin%
\pgfsetlinewidth{0.803000pt}%
\definecolor{currentstroke}{rgb}{0.000000,0.000000,1.000000}%
\pgfsetstrokecolor{currentstroke}%
\pgfsetdash{}{0pt}%
\pgfpathmoveto{\pgfqpoint{3.972882in}{0.387222in}}%
\pgfpathlineto{\pgfqpoint{3.972882in}{1.874182in}}%
\pgfusepath{stroke}%
\end{pgfscope}%
\begin{pgfscope}%
\pgfsetrectcap%
\pgfsetmiterjoin%
\pgfsetlinewidth{0.803000pt}%
\definecolor{currentstroke}{rgb}{0.000000,0.000000,0.000000}%
\pgfsetstrokecolor{currentstroke}%
\pgfsetdash{}{0pt}%
\pgfpathmoveto{\pgfqpoint{0.706528in}{0.387222in}}%
\pgfpathlineto{\pgfqpoint{3.972882in}{0.387222in}}%
\pgfusepath{stroke}%
\end{pgfscope}%
\begin{pgfscope}%
\pgfsetrectcap%
\pgfsetmiterjoin%
\pgfsetlinewidth{0.803000pt}%
\definecolor{currentstroke}{rgb}{0.000000,0.000000,0.000000}%
\pgfsetstrokecolor{currentstroke}%
\pgfsetdash{}{0pt}%
\pgfpathmoveto{\pgfqpoint{0.706528in}{1.874182in}}%
\pgfpathlineto{\pgfqpoint{3.972882in}{1.874182in}}%
\pgfusepath{stroke}%
\end{pgfscope}%
\end{pgfpicture}%
\makeatother%
\endgroup%
      
 \caption{State trajectory plot created with \mpl}
 \label{fig:Test}
\end{figure} 


\newpage
\section{Animation using \mpl}
\label{sec:animation}
Plotting the state trajectory is often sufficient, but sometimes it can be helpful to have a visiual represantation of the system to get a better understanding of what is actually happening. This applies especially for mechanical systems.
\mpl provides the sub-package \emph{animation}, which can be used for such a purpose. We therefore need to add 
\begin{lstlisting}
from matplotlib import animation
\end{lstlisting}
to the top of our code. Then we create a figure with some plot objects
\begin{lstlisting}
dx = 1.5*prmtrs.l
dy = 1.5*prmtrs.l
fig2, ax = plt.subplots()
    
# set axes limit, 
ax.set_xlim([min(min(x_traj[:, 0] - dx), -dx), 
         max(max(x_traj[:, 0] + dx), dx)])
ax.set_ylim([min(min(x_traj[:, 1] - dy), -dy), 
         max(max(x_traj[:, 1] + dy), dy)])
ax.set_aspect('equal')
ax.set_xlabel(r'$y_1$')
ax.set_ylabel(r'$y_2$')

# reference trajectory in the y1-y2-plane
x_ref_plot, = ax.plot([], [], 'r') 

# state trajectory in the y1-y2-plane
x_traj_plot, = ax.plot([], [], 'b') 
car, = ax.plot([], [], 'k', lw=2) # car
\end{lstlisting}
Now we want to display a representation of our car in the figure. We do this by plotting lines. All lines that represent the car are defined by points, which depend on the current state $\mathbf{x}$ and control signal $\mathbf{u}$. This means we need to define a function that maps from $\mathbf{x}$ and $\mathbf{u}$ to a set of points in the $y_1-y_2$-plane using simple geometry and passes these to the plot instance \emph{car}.
\begin{lstlisting}
def car_plot(x, u):
    """Mapping from state x and action u to the position of the car elements

    Args:
        x: state vector
        u: action vector

    Returns:
        car:

    """
    wl_length = 0.1 * prmtrs.l # wheel length
    y1, y2, theta = x
    v, phi = u

    # define chassis lines
    chassis_y1 = [y1, y1 + prmtrs.l * cos(theta)]
    chassis_y2 = [y2, y2 + prmtrs.l * sin(theta)]

    # define lines for the front and rear axle
    rear_ax_y1 = [y1 + prmtrs.w * sin(theta), y1 - prmtrs.w * sin(theta)]
    rear_ax_y2 = [y2 - prmtrs.w * cos(theta), y2 + prmtrs.w * cos(theta)]
    front_ax_y1 = [chassis_y1[1] + prmtrs.w * sin(theta + phi),
                   chassis_y1[1] - prmtrs.w * sin(theta + phi)]
    front_ax_y2 = [chassis_y2[1] - prmtrs.w * cos(theta + phi),
                   chassis_y2[1] + prmtrs.w * cos(theta + phi)]

    # define wheel lines
    rear_l_wl_y1 = [rear_ax_y1[1] + wl_length * cos(theta),
                    rear_ax_y1[1] - wl_length * cos(theta)]
    rear_l_wl_y2 = [rear_ax_y2[1] + wl_length * sin(theta),
                    rear_ax_y2[1] - wl_length * sin(theta)]
    rear_r_wl_y1 = [rear_ax_y1[0] + wl_length * cos(theta),
                    rear_ax_y1[0] - wl_length * cos(theta)]
    rear_r_wl_y2 = [rear_ax_y2[0] + wl_length * sin(theta),
                    rear_ax_y2[0] - wl_length * sin(theta)]
    front_l_wl_y1 = [front_ax_y1[1] + wl_length * cos(theta + phi),
                     front_ax_y1[1] - wl_length * cos(theta + phi)]
    front_l_wl_y2 = [front_ax_y2[1] + wl_length * sin(theta + phi),
                     front_ax_y2[1] - wl_length * sin(theta + phi)]
    front_r_wl_y1 = [front_ax_y1[0] + wl_length * cos(theta + phi),
                     front_ax_y1[0] - wl_length * cos(theta + phi)]
    front_r_wl_y2 = [front_ax_y2[0] + wl_length * sin(theta + phi),
                     front_ax_y2[0] - wl_length * sin(theta + phi)]
                     
    # empty value (to disconnect points)
    empty = [np.nan, np.nan]
   
    # concatenate set of coordinates
    data_y1 = [rear_ax_y1, empty, front_ax_y1, empty ,chassis_y1,
               empty, rear_l_wl_y1, empty, rear_r_wl_y1,
               empty, front_l_wl_y1, empty, front_r_wl_y1]
    data_y2 = [rear_ax_y2, empty, front_ax_y2, empty, chassis_y2,
               empty, rear_l_wl_y2, empty, rear_r_wl_y2,
               empty, front_l_wl_y2, empty, front_r_wl_y2]

    # set data
    car.set_data(data_y1, data_y2)
    return car,
\end{lstlisting}
For the animation to work we need to define another two functions, \emph{init()} and \emph{animate(i)}. The \emph{init()}-function defines which objects change during the animation.
\begin{lstlisting}
def init():  # only required for blitting to give a clean slate.
    x_ref_plot.set_data([], [])
    x_traj_plot.set_data([], [])
    car.set_data([], [])
    return x_ref_plot,
\end{lstlisting}
The \emph{animate(i)}-function assigns data to the changing objects, in our case the car and trajectory plots and the simulation time.
\begin{lstlisting}
def animate(i):
    """

    Args:
        i:

    Returns:

    """
    k = i % len(tt)
    ax.set_title('Time (s): ' + str(tt[k]), loc='left')
    x_ref_plot.set_data([], [])
    x_traj_plot.set_xdata(x_traj[0:k, 0])
    x_traj_plot.set_ydata(x_traj[0:k, 1])
    car_plot(x_traj[k, :], control(x_traj[k, :], tt[k]))
    return x_ref_plot,
\end{lstlisting}
Finally we have to pass these functions and the figure we created to \emph{animation.FuncAnimation()}.
\begin{lstlisting}
# animate
ani = animation.FuncAnimation(fig2, animate, init_func=init, 
                              frames=len(tt)+1,
                              interval = dt * 1000, blit=True)
                              
# save animation to mp4-file
ani.save('animation.mp4', writer='ffmpeg', fps = 1/dt)

# show animation
plt.show()
\end{lstlisting}
Now we have all things set up to simulate our system and animate it. In the next tutorial you'll learn how to design a tracking controller for this system.
\begin{figure}[ht]
	\centering
	\includegraphics[width=0.7\textwidth]{img/animation}
	\caption{Car animation}
	\label{fig:animation}
\end{figure}
\end{document}
