\documentclass[a4paper,11pt,headinclude=true,headsepline,parskip=half,DIV=13]{scrartcl}

% font, style, etc.
\usepackage[utf8]{inputenc} % defines
\usepackage[automark]{scrlayer-scrpage}
\usepackage{csquotes}
\usepackage{xspace} % proper space after macros with 0 args

% mathematics
\usepackage{amsmath}
\usepackage{amssymb}

% figures, tables, etc.
\usepackage{hyperref} %
\usepackage{graphicx}
\usepackage{tikz}
\usepackage{pgf}
\usepackage{xcolor}
\usepackage{placeins} % -> floatbarrier
\usepackage{siunitx}  % -> handling of units
%\usepackage[printwatermark]{xwatermark}
%\newwatermark[allpages,color=red!50,angle=45,scale=1.8,xpos=0,ypos=0]{\textsf{DRAFT ONLY,NOT APPROVED}}

% code
\usepackage{listings}
\lstset{
language=Python, 
backgroundcolor = \color{light-gray},
basicstyle=\scriptsize\sffamily,
stringstyle=\color{orange},
breaklines=true,
numberstyle=\tiny\color{gray},
keywordstyle=\bfseries\color{dark-blue}\textit, % print keywords dark-blue
commentstyle=\color{dark-green}, % print comments dark-green
showstringspaces=false} % spacing between strings not showed

\newcommand{\listcode}[3]{\lstinputlisting[numbers=left,firstnumber=#1,firstline=#1,lastline=#2]{#3}}
\newcommand{\listcodeplot}[2]{\listcode{#1}{#2}{../sim/01_car_example_plotting.py}}
\newcommand{\listcodeanim}[2]{\listcode{#1}{#2}{../sim/02_car_example_animation.py}}

% others
\usepackage{acronym}
\usepackage{luacode}
\usepackage{soul}

% theorems
\newtheorem{defi}{Definition}[section]

% setup the appearance of links
\hypersetup{
    colorlinks = true, % false -> red box arround links (not very nice)
    linkcolor={blue!100!black},
    citecolor={blue!100!black},
    urlcolor={blue!100!black},
}

% manage glossaries
% Call makeglossaries on a command prompt after LaTeX compiling,
% the re-run LaTeX
\usepackage{glossaries}
\setacronymstyle{long-short}
\makeglossaries
\newacronym{ivp}{IVP}{initial value problem}
\newacronym{ode}{ODE}{ordinary differential equation}

% define shortcuts
\newcommand{\ad}{\mathrm{ad}}
\renewcommand{\d}{\mathrm{d}} % d vor differential forms
\newcommand{\NV}{{\cal N}\,}
\newcommand{\rang}{\mathrm{rang}}
\newcommand{\im}{\mathrm{im}}
\newcommand{\spann}{\mathrm{span}}
\newcommand{\R}{\mathbb{R}} %  set of real numbers
\newcommand{\py}{\emph{Python}\xspace}
\newcommand{\scipy}{\emph{SciPy}\xspace}
\newcommand{\numpy}{\emph{NumPy}\xspace}
\newcommand{\mpl}{\emph{Matplotlib}\xspace}
\newcommand{\uu}{\mathbf{u}}
\newcommand{\f}{\mathbf{f}}
\newcommand{\x}{\mathbf{x}}
\newcommand{\y}{\mathbf{y}}
\newcommand{\z}{\mathbf{z}}
\newcommand{\xZero}{\mathbf{x}_0}

% color definitions
\definecolor{light-gray}{gray}{0.95}
\definecolor{dark-blue}{rgb}{0, 0, 0.5}
\definecolor{dark-red}{rgb}{0.5, 0, 0}
\definecolor{dark-green}{rgb}{0, 0.5, 0}
\definecolor{gray}{rgb}{0.5, 0.5, 0.5}

% Avoid ugly indentations in footnotes.
\deffootnote[1em]{1em}{0em}{%
\textsuperscript{\thefootnotemark}%
}

\luadirect{dofile("luainputlisting.lua")}
\newcommand*\luainputlisting[2]{
    \luadirect{print_listing(\luastring{#1}, \luastring{#2})}
}

% ----------------------------------------------------------------------------
\subject{\py for simulation, animation and control}
\title{Design and Simulation of LQR Control}
\subtitle{An introductory tutorial for the design and implementation of LQR controllers for time-invariant and time-variant linear systems}
\author{Robert Heedt\thanks{Institute for Control Theory, Faculty of Electrical and Computer Engineering, Technische Universität Dresden, Germany} \and Jan Winkler\footnotemark[1]}
\publishers{}
\date{\today}
% ----------------------------------------------------------------------------

% Headings
\pagestyle{scrheadings}
\ihead{\leftmark}
\chead{}
\ohead{Page \pagemark}
\ifoot{}
\cfoot{Python Control Tutorial 4}
\ofoot{}

\begin{document}

\maketitle




\tableofcontents

\newpage

\section{Introduction}
The goal of this tutorial is to teach the usage of the programming language \py as a tool for developing and simulating control systems.
The following topics are covered:
\begin{itemize}
    \item Flatness based feedforward control using existing trajectory generators
    \item Feedback control using LQR for linear time-invariant (LTI) system
    \item Demonstration of problematic situations
    \item Feedback control using LQR linear time-variant (LTV) system
\end{itemize}
Later in this tutorial this process is applied to design control strategies for the cart-pole system from a previous tutorial.

\section{Implementing the System}
In order to demonstrate the design methods discussed in the following, a simple academic example from \hl{??} will be used.
The system state~$x=(x_1, x_2)$ has two components and the scalar input is called~$u$.
Written in state-space-form, the system equation then is:
\begin{equation}
\dot x = F(x, u) = 
\begin{pmatrix}
a \sin(x_2)\\
-x_1^2+u\\
\end{pmatrix}\,.
\label{eq:academic_example_ss}
\end{equation}
Later, the jacobians
\begin{equation}
A(x^*, u^*) := \left.\frac{\partial F}{\partial x}\right\vert_{(x^*, u^*)}= \begin{pmatrix}0 & a \cos(x^*_2)\\-2 x^*_1 & 0\end{pmatrix}
\end{equation}
and
\begin{equation}
B(x^*, u^*) := \left.\frac{\partial F}{\partial u}\right\vert_{(x^*, u^*)}= \begin{pmatrix}0 \\ 1\end{pmatrix}
\end{equation}
will also be needed.

Implementing this system in Python then simply means expressing these terms as functions containing these computations. 
\luainputlisting{../sim/01_lqr.py}{defsystem}
Notable are mainly the usage of NumPy arrays for all vectors and the \lstinline{Parameters} class, please refer to a previous tutorial if unsure about these aspects.

\section{Trajectory Planning and Feedforward}

Similar to previous tutorials, feedforward control design can be simplified significantly by exploiting the flatness property of this system.
Specifically, the flat output is~$y=x_1$.
Recall, this means a desired trajectory $t \mapsto x_1^*(t)$ can be freely chosen (as long as it is sufficiently often differentiable).
Then, the trajectories for all other system variables (states and inputs) is calculated analytically without integration.

To this effect, the first system equation in~\eqref{eq:academic_example_ss} is solved for~$x_2$, yielding
\begin{equation}
x_2 = \arcsin\left(\frac{\dot x_1}{a}\right)
\label{eq:flatness_x2}
\end{equation}
which also introduces the constraint~$\forall t: |\dot x_1(t)| \leq |a|$ to trajectory planning.

After differentiating~\eqref{eq:flatness_x2} w.\,r.\,t.\ time, resulting in
$$
\dot x_2 = \frac{\ddot x_1}{\sqrt{a^2-\dot x_1^2}}\,,
$$
a term for $u$ is obtained by solving the second component of~\eqref{eq:academic_example_ss}:
$$
u = \dot x_2 + x_1^2 =  \frac{\ddot x_1}{\sqrt{a^2-\dot x_1^2}} + x_1^2\, .
$$
For the \py implementation, the trajectory planner from a previous tutorial is reused to obtain a polynomial that transitions from~$y(t_0) = y_0$ to~$y(t_f) = y_f$.
This function is then evaluated at the time values stored in vector \lstinline{t_traj}.
\luainputlisting{../sim/01_lqr.py}{plantraj}
The previously derived formulas are then translated into NumPy operations to obtain the values for~$x_2$ and~$u$ at every time step.
\luainputlisting{../sim/01_lqr.py}{flatness}

\printglossaries

\end{document}

%%% Local Variables:
%%% mode: latex
%%% TeX-master: t
%%% End: